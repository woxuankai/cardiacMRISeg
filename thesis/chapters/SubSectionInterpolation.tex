% copy 多模态医学图像配准研究
\subsection{图像插值}%\label{SectionModality}
在配准过程中,相似性测度(Similarity Metric)通常是比较固定图像(fixed image)
和运动图像(moving image)之间对应点的灰度值。
当一个点通过某种变换,从一个空间映射到另一个空间时,
目标点的坐标通常不在网格点上。
在这种情况下,插值算法就需要用来估计目标点的灰度值。
插值方法影响着图像的平滑性,优化的搜索空间,以及总体的计算时间。
因为在一个优化周期之中,插值算法将被执行成千上万次。
所以在指定插值方案时,我们需要在计算复杂性和图像平滑性之间做一个权衡。
图像插值技术可以分为两类,一类是基于确定性的方法,另一类是基于统计的方法。
它们之间的不同之处在于基于确定性的插值技术假定在采样点之间有某种确定的变异性,
比如线性插值中的线性;
基于统计的插值方法是以图像采样点的某种统计分布的评估误差最小化为基础的,
而且基于统计的方法计算效率不高。
基于确定性的插值算法的数值精度和计算代价直接依赖于插值核函数。
下面我们介绍几种常用的基于确定性的插值算法及其插值核函数。
\subsubsection{最邻近插值}
最邻近插值又称零阶插值,一维最邻近插值算法的数学表达式为:
\begin{equation}
  f(x)=\sum_{k=0}^1f(x_k)w(t)
\end{equation}
下式为插值核函数,其中$t=(x-X_k)/(x_1-x_0)$,
$X_0$,$X_1$,$X_2$为等间隔的已知采样点,$X$为待求插值点。
\begin{equation}
  w(t)=
  \begin{cases}
    1& 0\le\lvert t\rvert<0.5\\
    0& 0.5\le\lvert t\rvert
  \end{cases}
\end{equation}

二维图像插值需要对插值点进行两次一维插值。
它输出的像素值等于距离它映射到的位置最近的输入像素值。
对于二维图像,该法是取待采样点周围4个相邻像素点中
距离最近的1个邻点的灰度值作为该点的灰度值。
最邻近插值算法是一种简单的插值算法,计算速度非常快,
其不足是会使细线状目标边界产生锯齿。

\subsubsection{双线性插值}
线性插值又称为一阶插值,它的效果好于最邻近插值算法,
只是程序相对复杂一些,运行时间稍长些。
一维的线性插值和最邻近插值有相同的数学表达式,但插值核函数不同,
下式为线性插值的核函数,t的意义与最邻近插值中相同。
\begin{equation}
  w(t)=
  \begin{cases}
    1-\lvert t\rvert & 0\le\lvert t\rvert<1\\
    0& 1\le\lvert t\rvert
  \end{cases}
\end{equation}

对二维图像进行插值时,它先对水平方向上进行一阶线性插值,
然后再对垂直方向上进行一阶线性插值,而不是同时在两个方向上呈线性,
或者反过来,最后将两者合并起来,所以称为双线性插值。
这种方法是利用周围4个邻点的灰度值在两个方向上作线性内插
以得到待采样点的灰度值,
即根据待采样点与相邻点的距离确定相应的权值计算出待采样点的灰度值。
设$0<x<1$,$0<y<1$。
首先可以通过一阶线性插值得出$f(x,0)$:
\begin{equation}
  f(x,0)=f(0,0)+x(f(1,0)-f(0,0))
\end{equation}
类似地,对$f(x,1)$进行一阶线性插值:
\begin{equation}
  f(x,1)=f(0,1)+x(f(1,1,)-f(0,1))
\end{equation}
最后,对垂直方向进行一阶线性插值:
\begin{equation}
  f(x,y)=f(x,0)+y(f(x,1)-f(x,0))
\end{equation}
合并上述三式得:
\begin{equation}
  \begin{split}
    f(x,y)=&(f(1,0)-f(0,0))x+(f(0,1)-f(0,0))y\\
    &+(f(1,1)+f(0,0)-f(0,1)-f(1,0))xy+f(0,0)
  \end{split}
\end{equation}

一般情况下,在程序中进行双线性插值计算时直接使用3次一阶线性插值即可,
这样只要计算3次乘法运算和6次加法运算,
而用合并后的式子需要计算4次乘法运算和8次加法运算。
上面的推导是在单位正方形上进行的,可以推广到一般情况中使用。

\subsubsection{双三次插值}
双三次插值又称立方卷积插值,是一种三阶插值方法,
即不仅考虑到四个直接邻点灰度值的影响,还考虑到各邻点间灰度值变化率的影响
,利用了待采样点周围更大邻域内像素的灰度值作三次插值。
因为是在两个方向分别进行三次插值,所以又称为双三次插值。
一维情况下,设$X_{-1}$,$X_0$,$X_1$,$X_2$为等间隔的已知采样点,
$X$为待求插值点,设$h=X_1-X_0$,$s=X-X_k$,令$t=s/h$,
则所求插值点的表达式为:
\begin{equation}
  f(x)=\sum_{k=-1}^2f(x_k)w(t)
\end{equation}

三次插值函数核函数表达式为:
\begin{equation}
  w(t)=
  \begin{cases}
    (a+2)\lvert t\rvert^3-(a+3)\lvert t\rvert^2+1& 0\le\lvert t\rvert<1\\
    a\lvert t\rvert^3-5a\lvert t\rvert^2+8a\lvert t\rvert-4a
    &1\le\lvert t\rvert<2\\
    0& 2\le\lvert t\rvert
  \end{cases}
\end{equation}

对于二维图像,插值算法要在二维方向应用上述算法。
假设所求插值像素点为$F$,先在水平方向对每一行依据上述插值原理插值,
得到4个临时插值像素点的像素值$F_{-1}$,$F_0$,$F_1$,$F_2$,
然后在垂直方向对这四个点用上述算法进行插值求得所需插值点$F$的值。
插值的效果,取决于参数a和图像的频
域性质,一般取a的值为$-1$,$-0.75$,$-0.5$。
针对不同的图像来选择$a$的值以获得最好的插值效果。

%\subsubsection{B样条插值}
%\subsubsection{窗函数插值}
