% !Mode:: "TeX:UTF-8"

\chapter{绪论}
心血管疾病(Cardiovascular diseases, CVDs)在世界范围仍然是最大的死亡原因之一。
在2008年,超过一千七百万人死于心血管疾病。
令人遗憾的是,在这些死者中,超过三百万人在死亡时不超过六十岁,
而这一切,很大程度上是可以被预防的。
因心血管疾病导致的过早死亡的比例在高收入国家超过4\%,在低收入国家超过42\%,
这导致了在国家和人口之间心血管疾病的发生率的日趋严重不平等。
\citeup{WHOCardiovascularDisease2011}

有新的层面的问题在这个令人震惊的情况发生。
在过去的二十年中,因心血管疾病导致的死亡人数在高收入国家减少,
但在中低收入国家以惊人的速度增加。\citeup{WHOCardiovascularDisease2011}

Non-invasive assessment of left ventricular function is an important part of the
diagnosis and management of cardiovascular disease. Cine MRI has been proven
to be an accurate and reproducible modality for quantitative evaluation of left
ventricular function [1–6]. Relevant measurements include ventricular volume,
mass, and cavity ejection fraction (EF), which are based on the results of de-
lineation of endocardial and epicardial boundaries by segmentation techniques.
However, LV segmentation is still a open problem and is challenging due to poor
contrast between tissues around the epicardium and intensity inhomogeneities
in cine CMR images.


心血管疾病已经成为人类健康的头号杀手。迅速发展的医学成像技术,给心血管
疾病诊断带来新希望的同时,也给传统诊断方式带来巨大的阅片负担。借助计算机处
理技术,研究心脏组织分割与可视化关键算法意义重大。本章首先讲述本文的研究背
景与意义,结合研究现状和发展趋势,阐述面临的困难与挑战,进而介绍本文的研究
目的、内容以及创新点,最后给出本文的研究路线和组织结构。

\section{研究背景与意义}


\section{国内外研究现状}

通过连续观察一系列心脏影像,结合临床诊断经验由医生给出诊断报告的诊断模
式,使得诊断结果高度依赖于医生的主观经验,给心血管疾病的诊断带来很多不确定
性。特别地,时间分辨率和空间分辨率大幅提高的影像设备在增加影像精度的同时也
增加了临床医生的阅片负担。因此,借助计算机辅助诊断技术,处理分析心脏影像进
而诊断心血管疾病成为目前国内外研究热点【58】。然而,医学成像设备采集到的心脏影
像往往存在灰度不一致性和噪声。为了提高影像处理分析算法的精度,需要对心脏影
像进行偏移场校正和影像噪声消除处理。现有的偏移场校正算法可以归类为前处理方
法和后处理方法。前处理方法的典型代表包括采用放置灰度分布信息己知的物体模型、
多线圈采样、采用特殊的脉冲序列等来降低外界环境对影像灰度影响的成像方法[59-61】。
文献f621指出,后处理的偏移场校正方法主要包括:基于低通滤波模型的方法、超曲面
模型方法、统计模型方法、灰度信息模型方法、基于分割的方法。传统噪声消除方法
通常基于噪声是振荡的,正常组织的灰度是平滑的或分段平滑的假设,将组织微小细
节和噪声等同对待,消除影像噪声的同时也会扭曲影像中的结构信息,给医学影像数
据带来新伪影【631。近些年提出的Non.Local去噪方法有效消除影像噪声的同时,不会
扭曲心脏影像,已经成为当前研究热点【631。经过偏移场校正和噪声消除等影像预处理
后的心脏影像影像有助于提高影像处理分析方法的精度。影像分割是影像处理分析关
键技术之一【641。通常地,分割方法通过影像灰度、形状模型、配准或分类的方法从医
学影像中提取感兴趣区域。由分割结果可以计算并获取心脏全局参数和各种运动参数
以辅助医生进行心血管疾病的早期诊断。心脏可视化是影像处理分析另一关键技术【65】。
该技术可以定性和/或定量建立心脏三维、四维模型,并通过不同的绘制技术将实时跳
动的心脏和分割到的组织器官以不同的颜色渲染出来,且可以显示心脏功能参数信息。


由左心室分割算法定位出心室内膜和外膜后的分割结果可以估计心室体积、心肌
质量和射血比例等左心室功能参数。但是受心室外膜附近糟糕对比度和整个心脏影像
灰度不均匀性的影响,左心室分割仍然是一个开放性的具有挑战的问题【661。手动分割
是最显而易见的左心室分割方法【6‘n。但是,由经过特殊训练的内科医生或技术人员手
动勾勒出左心室分割曲线是一项耗时且乏味的工作。另外,手动分割往往在勾画者内
和勾画者间存在差异。也就是说,勾画者有自己的喜好和倾向。同时,同一勾画者在
不时间对同一影像描绘的分割曲线也不完全相同。除此之外,手动分割一层一层地描
绘左心室内外膜曲线,并没有考虑到心室的空间结构信息。为了克服手动分割的缺点,
领域内研究人员提出了许多自动或半自动的左心室分割方法168-76】。传统的分割技术像
阈值、分类、聚类都被用于左心室分割【6引。活动形状模型(Active
 Shape Model,ASM)
 及由其演化来的活动外观模型(Active
  Appearance
   Models,AAMs)已经成为杰出的左
   心室分割方法[69-71】。但是这类方法严重依赖训练数据集的规模和类别丰富程度。基于
   配准的左心室分割方法需要手动分割数据集做为模版,且算法分割精度受配准精度限
   制【72】。由于可以提供光滑的封闭曲线作为分割结果且可以达到亚像素分割精度,活动
   轮廓模型和水平集方法广泛应用于影像分割领域173-76】。文献[73】提出一种耦合的表面蔓
   延方法来分割左心室心肌。该方法通过控制内膜曲线和外膜曲线将心肌厚度限定在给

定区间。显然,这个限制并不符合心肌的解剖结构。Paragios和Lynch进一步改进了这
种方法[74,75】。他们用两个不同水平集函数的零水平分别表示心室内膜和外膜。两个水
平集函数耦合在一起演化。同样地,两个零水平集之间的距离在迭代过程中被限定在
指定区间。2005年Chung和Vese提出一种多层水平集方法(76】。该方法用同一水平集
函数的多个水平集表达影像对象的多个边界。但是,他们并没有验证这种方法用于心
室分割的有效性,并且该方法并没有在代表影像边界的多个水平集之间增加任何条件
约束。以上提到的方法均不具备处理影像灰度不一致性的能力。2011年发表在Medical
Image
 Analysis上的综述性文章[77】总结了发表于1993到201 1年之间的70篇关于左心
 室分割的文章,并以是否需要先验知识来将左心室分割算法分为两类。医学影像处理
 分析领域顶级会议MICCAI(Medical
  Image Computing
   and Computer—Assisted
   Intervention),于2009年组织了名为“Cardiac MR
    Left
     Ventricle Segmentation
     Challenge”的比赛【78】。该比赛只评价分割算法在收缩末期和舒张末期的分割结果。最
     终出版的会议录收集了8各分割算法,但各算法的分割结果都差强人意。
     
2613年,该
会议又组织了名为“MICCAI
Challenge Workshop on Segmentation:Algorithms,Theory and Applications”的比赛,
其中左心室分割是该workshop的一个重要组成部分【79】。
与 2009年的比赛不同,此次比赛评价整个心跳周期内左心室的分割结果。参赛的9种分
割方法按是否需要配准算法支持被分为两类。

\section{本文研究内容与创新点}

\section{本文组织结构}
