% !Mode:: "TeX:UTF-8"

\chapter{绪论}


\section{研究背景与意义}

心血管疾病(Cardiovascular diseases, CVDs)在世界范围仍然是最大的死亡原因之一。
在2008年,超过一千七百万人死于心血管疾病。
令人遗憾的是,在这些死者中,超过三百万人在死亡时不超过六十岁,
而这一切,很大程度上是可以被预防的。
因心血管疾病导致的过早死亡的比例在高收入国家超过4\%,在低收入国家超过42\%,
这导致了在国家和人口之间心血管疾病的发生率的日趋严重不平等。
有新的层面的问题在这个令人震惊的情况发生。
在过去的二十年中,因心血管疾病导致的死亡人数在高收入国家减少,
但在中低收入国家以惊人的速度增加。\citeup{WHOCardiovascularDisease2011}

对于左心室的非入侵式(non-invasive)的评价
是对于心血管疾病的诊断与病症掌握的重要部分。
电影式磁共振成像(cine MRI)已经被证明是一种能定量估计左心室功能的、
准确并且可重现的形态。
相关的估计包括了心室体积,质量和腔体射血分数(ejection fraction, EF),
这些估计都是基于对心内膜(endocardium)与心外膜(epicardium)界限的分割技术的结果。
\citeup{LevelSetMICCAI2013}
%图\ref{CineMRIPresentation}\ref{S1}展示了一次电影式磁共振成像结果的部分结果,
%图\ref{ManualSegOfCineMRI}展示了扫描结果在舒张末期的三维图像的手工分割结果
%\footnote{完整的扫描结果是以三维空间$+$时间(3D$+$T)的形式保存的}
%\footnote{图片来源:MICCAI Challenge Workshop on Segmentation: %
%Algorithms, Theory and Applications ("SATA")}
%\pic[hb]{舒张末期专家分割结果}{width=0.5\textwidth}{ManualSegOfCineMRI}

%\begin{pics}[h]{电影式磁共振扫描结果(部分)\newline}{CineMRIPresentation}
%  \addsubpic{t0-z0}{width=0.14\textwidth}{S1}
%  \addsubpic{t0-z2}{width=0.14\textwidth}{S2}
%  \addsubpic{t0-z4}{width=0.14\textwidth}{S3}
%  \addsubpic{t0-z6}{width=0.14\textwidth}{S4}
%  \addsubpic{t0-z8}{width=0.14\textwidth}{S5}
%  \addsubpic{t0-z10}{width=0.14\textwidth}{S6}\newline
%  \addsubpic{t0-z8}{width=0.14\textwidth}{V1}
%  \addsubpic{t3-z8}{width=0.14\textwidth}{V2}
%  \addsubpic{t6-z8}{width=0.14\textwidth}{V3}
%  \addsubpic{t9-z8}{width=0.14\textwidth}{V4}
%  \addsubpic{t12-z8}{width=0.14\textwidth}{V5}
%  \addsubpic{t15-z8}{width=0.14\textwidth}{V6}
%\end{pics}

快速发展的影像技术带来了海量的影像,也给影像科医生带来了巨大的工作量。
对于电影式磁共振成像得到的的四维影像(三维空间$+$时间)的全手动分割是不现实的,
因为这将浪费一名经验丰富的医师大概两个小时来分割一个典型的有150个切片的数据集。
因此,一般只将完整数据集中的舒张末期和收缩末期的图像进行手动分割。
如果想利用所有得到的心脏磁共振图像来进行量化分析,我们需要全自动的分割方法。
除此之外,手动分割还有一个问题,分割结果是受主观影响的:
同一名医师的不同次分割结果、不同医师的分割结果,都存在差异。
自动化的分割让不同时间、不同地点的分割结果更容易比较,
因为自动话的分割结果是完全定义好的、确定性的过程。
最后,对于一些比较好的方法,无论扫描的方向如何,自动化的分割结果都是一致的,
这对于手动分割来说,是个非常乏味的过程。\citeup{KedenburgAutomatic2006}


\section{多图谱分割方法简介}
多图谱分割方法(Multi-atlas segmentation, MAS)是
首先由Rohlfing, Klein, Heckemann等人创新性的工作
\citeup{Rohlfing2004,Klein2005,Heckemann2006}中提出并推广的一种图像分割方法。
经过多年发展,多图谱分割方法正在逐渐成为生物医学领域应用最广、最成功的
图像分割技术之一\citeup{MultiAtlasSurvey}。
通过操纵并使用整个``图谱''(图谱指的是已经事先被专家等手工标记好的训练数据)
数据集,而不是一些基于模型的平均表示,
多图谱方法在捕捉不同解剖结构方面有更好的灵活性,因此也提供了更卓越的分割精度。
除此以外,近来计算机硬件和图像处理软件的快速发展在这个问题的解决上提供了帮助,
也为多图谱分割方法的广泛采用提供了便利。
今天,多图谱分割方法已经走出了一条长路,
包括了一系列从机器学习、概率建模、优化、计算机视觉等众多领域获得灵感的算法。
明天,我们相信,多图谱分割方法会成为生物医学图像分割领域的主要方法之一。
\citeup{MultiAtlasSurvey}


\section{国内外研究现状}

尽管非常重要,右心室的全自动分割仍然是一个尚未解决的并且具有挑战性的问题
\citeup{LevelSetMICCAI2013}。
一方面,在电影式磁共振成像中,由于对成像时间有要求,心脏长轴方向分辨率较低,
成像结果的场偏差和部分容积效严重且噪声大,
心脏的运动、血液流动也对成像结果有一定干扰;
另一方面,心脏的结构复杂且心内膜周围的组织对比度并不高,难以区分,
而且心脏在一个心动周期内的形变非常大。
这些因素导致许多方法很难达到好的效果。

为了解决这一问题,研究人员提出了许多方法,如传统的聚类、分类、阈值等,
另外,活动形状模型(Actice Shape Model, ASM, 又称snake)
和水平集方法(Level Set)已经被证明对左心室分割有非常好的效果
\citeup{PetitjeanReview2011}。
医学图像处理领域的顶级会议,医学图像计算与计算机辅助干预协会
(The Medical Image Computing and Computer Assisted Intervention Society,
The MICCAI Sociery)分别在2009和2013年组织了名为
``Cardiac MR Left Ventricle Segmentation Challenge''和
``MICCAI Challenge Workshop on Segmentation:
Algorithms, Theory and Applications (SATA)''的比赛。

\section{本文研究内容与创新点}

本文将在大脑图像分割中大获成功的多谱图分割方法用于左心室分割。
使用MICCAI 2013年比赛中提供的数据,通过非刚性形变配准图谱,
通过互信息量和数据本身提点进行图谱选择,使用多数投票法融合分割结果,
并对得到的结果进行评价。
除此之外,考虑到多图谱方法在图像配准中计算量较大的问题,
针对MICCAI 2013的数据集,本文还在图谱选择方面作出一些改进,
极大减少了运算时间。

\section{本文组织结构}
本文共四章。

第一章介绍了问题的提出与该课题的意义。

第二章首先概述了传统的分割方法,
然后从配准、图谱选择、标签融合等方面着重介绍了多图谱分割方法
并简述了分割结果的评价方法。

第三章从多方面描述了本人所做的工作,并对结果做了评价与分析。

最后一章对本文提出的方法做了归纳,并总结了该方法的意义、改进方向与展望。

