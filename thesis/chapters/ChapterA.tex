% !Mode:: "TeX:UTF-8"

\chapter{绪论}
Cardiovascular diseases (CVDs) remain the biggest cause of
deaths worldwide. More than 17 million people died from CVDs
in 2008. More than 3 million of these deaths occurred before the
age of 60 and could have largely been prevented. The percentage
of premature deaths from CVDs ranges from 4\% in high-income
countries to 42\% in low-income countries, leading to growing
inequalities in the occurrence and outcome of CVDs between
countries and populations.\citeup{WHOCardiovascularDisease2011}

There are also new dimensions to this alarming situation. Over
the past two decades, deaths from CVDs have been declining in
high-income countries, but have increased at an astonishingly
fast rate in low- and middle-income countries (LMIC).\citeup{WHOCardiovascularDisease2011}

心血管疾病已经成为人类健康的头号杀手。迅速发展的医学成像技术,给心血管
疾病诊断带来新希望的同时,也给传统诊断方式带来巨大的阅片负担。借助计算机处
理技术,研究心脏组织分割与可视化关键算法意义重大。本章首先讲述本文的研究背
景与意义,结合研究现状和发展趋势,阐述面临的困难与挑战,进而介绍本文的研究
目的、内容以及创新点,最后给出本文的研究路线和组织结构。

\section{研究背景与意义}
随着生活水平的提高和人类预期寿命的延长,心血管疾病(CardioVascular
 Disease,
 CVD)已经成为人类健康的头号杀手【 ̈。作为世界范围内发病率和死亡率最高的疾病,
 心血管疾病是一类涉及心脏、血管,或两者兼而有之的疾病【1】。尽管20世纪70年代以
 来,心血管疾病死亡率在许多发达国家有所下降,但是这类疾病正以很快的速度在中
 低等收入国家发展121。世界卫生组织(World
 Health  Organization,WHO)公布数据显
   示,2008年全球约有1730万人死于心血管疾病,约占死亡总数的30\%【3】。其中,中低
   等收入国家死于心血管疾病的人数占世界同类死亡人数的80\%还要多,且接近占其国
   内总死亡人数的30\%【31。预计到2030年,每年将有超过2300万人死于心血管疾病[31。

\subsection{心脏的解剖结构}
为了很好地了解心血管疾病的病症与病因,进而制定合理有效的应对策略,有必
要了解心脏的解剖结构。如图1.1所示,整个心脏主要有两房两室构成。胸肋面,即心
脏的前面,主要由右心室和小部份左心室组成。隔面,也就是器官与横隔膜接触的那
一面,主要有左心室组成。整个心尖均由左心室组成。为了以较高的压力把血液泵出,
与右心室相比,左心室的肌肉更厚更发达。由图1.1可知,心肌(myocardium)介于心
室内膜(endocardium)和外膜(epicardium)之间。与右心室有三个乳头肌(papillary
muscle)通过腱索(chordae tendineae)牵引控制三尖瓣开关不同,左心室只有两个乳
头肌通过腱索控制二尖瓣。乳头肌的另一端通过心肉柱(trabeculae
 carneae
  or
  trabeculation)连接在心室内壁上。右心室(right ventricle)舒张时,经体循环由上下
  腔静脉(superior
   and inferior
    vena
     cava)流回右心房(right atrium)的缺氧血液通过三
     尖瓣(tricuspid valve)涌入右心室。右心室收缩时,三尖瓣关闭防止血液倒流回右心
     房。与时同时,肺动脉瓣(pulmonary valve)打开,右心室将缺氧血液压入左右肺动脉
     (1eft and fight pulmonary
      artery)进入肺循环。左心室(1eft ventricle)舒张时,经肺循
环由左右肺静脉(1eft and right pulmonary
veins)流回左心房(1eft amum)的富含氧气
的新鲜血液,通过二尖瓣(mitral valve)涌入左心室。左心室收缩时,二尖瓣关闭主动
脉瓣(aortic valve)打开,左心室将富含氧气的血液压入主动脉(aorta)进入体循环。
显然,心室功能特别是左心室功能在人体血液循环中起着至关重要的作用。

不停搏动的心脏为人体器官提供富含营养的血液,用于维持它们正常的生命活动。
作为供血的动力器官,心脏本身的营养供给是由流通于冠状动脉内部的血液来保证的。
冠动脉起于主动脉(aorta)根部,分左右两支,行于心脏表面【41。如图1.2所示,心脏
如一倒置的、前后略扁的圆锥体。如将其视为头部,则位于头顶部、几乎环绕心脏一
周的动脉恰似一顶王冠,这也是将其称为冠状动脉的原因。解剖学上,将源于升主动
脉(aorta ascending)的两个冠状动脉分支分别称为左冠状动脉(Left
 Coronary Artery,
 LCA)和右冠状动脉(Right Coronary Artery,RCA)。左冠状动脉起于主动脉的左冠状
 动脉窦,向左行于左心耳与肺动脉干之间。如图1.2所示,左冠状动脉主干(Left Main,
 LM)通常包含左前降支(Left
  Anterior
   Descending,LAD)和左回旋支(Left
   CircumfleX,LCX)两个分支。极少数人在LAD和LCX分支之间还有第三个分支
   (intermediate)。左前降支为左冠状动脉的直接延续,沿前室问沟下行,又分成左冠
   状动脉斜行支(Diagonal branches,Dx)和室间隔支(Septal branches,Sx)。人类心
   脏通常有一条或多条左冠状动脉斜行支。左回旋支从左冠状动脉主干发出后即行走于
   左侧冠状沟内,绕心左缘至左心室膈面,多在心左缘与后室间沟之间的中点附近分成
多个锐缘支(Marginal Branches,Mx)而终。右冠状动脉起于主动脉的右冠状动脉窦,
行于右心耳与肺动脉干之间,再沿冠状沟右行,绕心下缘至膈面的冠状沟内。一般在
房室交点附近或右侧,分为多条短小的右侧圆锥支、锐缘支(Acute Marginal
 branch,
 AM)、后降支(Posterior
  Descending
   Artery,PDA)和房室结支(AtrioVentricular
   node,AVnode)。


\subsection{心脏成像与心血管疾病诊断}
心血管疾病是包含心肌性疾病、心内膜炎、・bl-膜炎、心室肥大、冠心病等在内
的心脏和/或血管类疾病[51。
心室肥大(ventricular hypertrophy)是主要的心血管疾病之一。在众多诱因中,高
血压是引起心室肥大的主要因素之一【61。这是由于高血压患者的心脏需要提供更大的压
力来给人体供血,久而久之心室肌肉组织变的越来越厚。如图1.3所示,心室肌肉变厚
必然会挤压心室容量。而心室容量的减小将导致心脏射血分数降低,从而影响人体组
织的正常血液供给【6】。另一方面,心室肥大还是冠心病的独立危险因素。这是因为增厚
的心肌组织必然增加心脏的自身供血负担,从而增加人体患冠心病的风险。长期的左
心室肥大容易导致心力衰竭,甚至引起心脏病突发死亡【7】。临床上,心室功能估计与定
量分析有助于医生发现心室肥大,并制定合理有效的治疗方案。
与心室肥大对应,冠心病(coronary
 heart
  disease)是另一类常见的心血管疾病【8】。
  它是一种由冠状动脉狭窄或阻塞引起的心肌缺血缺氧或心肌坏死类心脏病。冠状动脉
  狭窄多系脂肪物质沿血管内壁堆积所致。随堆积物增加,逐渐加重的冠状动脉狭窄会
  影响心肌的血液供应。由于心脏得不到足够的氧气供给,冠心病患者就会发生心肌缺
  血缺氧性心绞痛。当冠状动脉狭窄发展到一定程度,部分冠状动脉会发生阻塞,从而
  严重影响心肌供氧,使得部分心肌因缺氧而坏死【引。如图1.4所示,为了减少心绞痛给
  患者带来的痛苦或救治因突发心肌梗塞而危及生命的患者,临床上主要采用支架术和
  搭桥术来治疗冠动脉狭窄【91。因此,准确定位冠状动脉狭窄,并分析其狭窄程度有助于
  医生制定有效的手术方案。
  早发现、早预防、早治疗可以有效降低心血管疾病给人类健康带来的威胁【10】。基
  于立体直观及测量方便的优势,经皮冠状动脉介入治疗(Percutaneous
   Coronary
   Interventions,PCI)已经成为冠心病、先天性心脏病及其他心脏组织功能异常的主要
   诊断方法【...。这种诊断方法是有创检查,且具有一定的并发症风险。目前临床上广泛
   采用x线冠状动脉造影(Coronary Arteriography,CAG)来诊断冠心病,并进行介入
   性治疗【12】。X线CAG也是有创检查,且诊断费用较高。造影结果只能反映血管腔被造
   影剂充填的轮廓,不能反映血管壁结构和血管生理功能状况。国外文献报告冠状动脉
   造影常低估冠状动脉病变程度,漏诊率约为10\%【l 31。
伴随心脏病学研究的快速发展与医疗技术的进步,越来越多的成像技术用于诊断
心血管疾病【14】。这些技术利用无创伤方法准确、快速地将心脏结构成像,为进一步诊
断心脏疾病、评价心脏功能提供影像支持【 ̈】。为了能够实现心脏成像,理想的成像技
术应该能够(1)在心脏跳动过程中完成心脏组织扫描;
 (2)提供高分辨率的可以观
 察到详细心脏解剖结构的影像:
  (3)用于分析心脏功能参数,包括射血比例、心肌质
  量等;
   (4)在影像中展现冠状动脉软性狭窄、钙化狭窄或两者兼有的狭窄。随着成像
   速度和质量的提升,磁共振成像和计算机断层扫描被逐渐应用到心血管疾病检查中,
   并陆续成为检查心脏功能障碍和冠状动脉狭窄等相关疾病的标准成像方法【l引。

磁共振成像(Magnetic
 Resonance
  Imaging,MRI)是一种生物体原子核自旋断层
  成像技术【161。物理学上,在外界磁场与射频脉冲(Radio.Frequency Pulse,砌’P)共同
  作用下,生物体会发生磁共振(Magnetic Resonance,MR)现象。通过采集磁共振发
  出的电磁信号,MXI能够重建出包含人体组织结构信息的影像。M对无需注射造影剂
  即可成像人体组织,且扫描过程无电离辐射,对人体无不良影响。除此之外,磁共振
  成像还可以提供高分辨率、高对比度的任意重建平面的断层影像。因此,作为一种无
  损伤体外探测技术,该技术的出现推动了医学影像处理科学的发展,并使得磁共振成
  像成为2l世纪生物学和医学研究的重要工具之一【l 71。磁共振成像已经应用在许多领域,
  但是受心跳、弛豫时间和复杂的重建算法影响,磁共振成像在心脏成像领域的发展较
  为缓慢【17】。随着科技的进步,特别是快速成像序列和心电图门控(Electrocardiograph
  Gating,ECG.Gating)等相关技术出现之后,这些障碍被逐渐克服,进而促成了心血管
  磁共振成像的快速发展【18】。心血管磁共振成像,俗称心脏磁共振,是一种用于评估心
  血管功能的非侵入式医学成像技术。它遵循磁共振成像基本原理,针对心血管成像难
  点,优化现有的成像技术,使其应用于心血管疾病临床诊断中。心脏磁共振是在K空
  间(即频率域空间)轨迹的引导下采集心脏影像的[221。众多K空间轨迹中,
  PROPELLER是其中一种含“自导航”信息的中心过采样轨迹【231。基于此轨迹的磁共
  振成像方法可以有效抑制运动伪影,并能够提高心脏影像的信噪比和对比度【241。
  心脏磁共振成像面临的最大挑战是如何消除存在于影像扫描过程中的心脏跳动和
  呼吸运动在重建影像上产生的伪影【19】。呼吸伪影可以由患者在扫描过程中屏住呼吸得
  到有效减弱【201。而心跳伪影则可以通过心电图门控方法得到有效控制【21】。与传统的心
  电图门控成像方法在心脏基本不发生形变的心跳时间段采集心脏数据不同,作为一种
  可以成像实时跳动心脏的方法,电影(Cine)磁共振成像是一种可以用于评估心肌和
  瓣膜功能的心脏成像技术【251。该技术可以在一次屏气的一个R—R间期获得心脏的多个
  相位影像,从而实现实时追踪心跳周期内不同心肌位置的目的。通常地,心脏Cine
  MR将整个心跳周期分成若干段(帧),在每一段分别采集一幅心脏影像,每幅影像由
  多个心跳周期内同一心跳位置的结构信息构成。采集过程中,依照心率不同R波之间
  的分段情况也不尽相同。按时间顺序依次观察各帧影像时即得到一个跳动的心脏。基
  于可以提供心脏时间序列影像这一技术特点,Cine MXI已经广泛应用于研究心脏功能
  及心肌在心跳周期内的变化情况等等【26】。特别地,基于PROPELLER和心电图门控的
  优点,Cine MRI可以得到清晰的心脏影像序列f27】。图1.5分别从横断面、平行于室间
  隔的心脏长轴位、垂直于室间隔的心脏长轴位及垂直于室间隔的心脏短轴位给出心脏
  Cjne MRI影像。
  随着医学成像技术的进步,作为X射线成像技术的扩展,电子计算机断层扫描
  (Computed
   Tomography,CT)已经可以用于心血管疾病检查(四)。X射线成像是一种
   光线投影成像方法,只能显示一个方向上的物体,如果被检测物体内部存在不同物质
   的重叠,这种成像方法就不能提供准确的影像了。而CT是一种采用多方向x光照射,
   利用计算机技术对被测物体断层扫描并重建出相关影像的成像方法【291。通常地,CT扫
   描仪利用精确准直的x线束与灵敏度极高的探测器一同围绕人体的某一部位作一个接
   一个的断面扫描,扫描过程中由探测器接收穿过人体后衰减的x线信息,再由快速模/
   数转换器将模拟量转换成数字量。计算机收集这些信息,计算得出该断面各点的x线
   吸收数值,并将这些信息合成断面灰度影像。自从Ledley于1976年设计出第一套可以
   进行全身扫描的CT系统以来,CT成像技术在临床上获得了广泛应用【291。然而,受扫
   描速度和时间分辨率限制,早期CT不适用于心脏检查。虽然电子束CT(Eleetron
   Beam
    CT,EBCT)具有较快的扫描速度和较高的时间分辨率,能够满足心血管疾病的
    诊断要求,使无创性冠脉成像及心肌评价成为可能,但EBCT信躁比及空间分辨率较
    差、设备昂贵且技术不够完善,至今仍未能普及【30】。随着成像技术的不断发展,多排
    鹱
    CT(Multi.Detector Computed Tomography,MDCT)抑或多层CT C Multi.Slice
    Computed
     Tomogmphy,MSCT)的出现,开创了CT用于心血管疾病检查的新纪元。
     作为当前临床应用的前沿产品,MSCT具有扫描速度快、信息量大、影像清晰、操作
     简单、安全可靠、重复性好、价廉无创等优点f311。因此,自1998年出现以来,MSCT
     扫描技术的发展日新月异,扫描速度越来越快,扫描覆盖范围越来越宽,成像质量越
     来越高,临床应用越来越广132]。2008年,Toshiba公司推出了320排螺旋cT。
     强奔
     PHILIPHS公司于2009年推出了256层CT产品【33】。与传统的单层CT相比,MSCT利
     用可变速扫描技术和多扇区重建技术,使得其时间分辨率与空间分辨率显著提高,并
     大大提高了长轴方向的空间分辨率,实现了各向同性。配以心电图门控装置和先进的
     三维后处理功能,MSCT为显示冠状动脉开辟了良好的前景【341。利用其高空间分辨率、
     亚秒级扫描速度,MSCT在冠状动脉狭窄检查中,给影像科医生带来了新机遇,也给
     广大心血管疾病患者带来了希型”】。其中,冠状动脉CT成像(Computed
      Tomography
      Angiography,CTA)具备冠状动脉无创检查能力,可以为临床诊断冠心病提供有力支
      持【36】。CTA检测冠状动脉狭窄具有高敏感性和特异性,己经成为冠心病检查的标准方
      法之一【3。71。随着MSCT的快速发展,其在心血管疾病中的诊断价值越来越大。除了进
      行冠状动脉显影、支架、搭桥术后的评价之外,MSCT还可以通过ECG回顾性重建跳
      动的心脏,从而可以进行心脏功能如射血分数、左室舒张末容积、左室收缩末容积及
      心肌质量等参数的评价。这些功能参数可以帮助医生进行心脏疾病诊断、外科手术介
      入及药物定位治疗【38】。
      除此之外,作为一种简便、易行的诊断方法,超声也被应用于心脏及冠状动脉功
      能评估中‘391。然而,受操控者及血管邻近结构影响,超声测量的重复性及准确性还有
      待验证。血管内超声(Intravascular Ultrasound,IVUS)是诊断冠状动脉粥样硬化斑块
      (钙化狭窄)稳定性的主要手段之一,但该方法是有创检查且费用较高【矧。正电子发
      射断层成像(Positron
       Emission
        Tomogmphy,PET)可以显像心肌代谢情况,是目前判
        断心肌细胞活性最准确的方法【411。PET成像可以帮助医生识别坏死心肌,对冠状动脉
        搭桥手术有重要的指导作用【4 ̈。
        作为非侵入式检测方法,现代医学成像技术已经在人类生活中扮演着越来越重要
        的角色【42】。根据成像方式的不同,现有的成像方法可以归为包含M砒、CT、超声成像
        (Ultrasound
         Imaging)等在内的结构成像和包括功能磁共振成像(Functional Magnetic
 Resonance
          Imaging,fMRa)、PET和单光子发射计算机断层成像技术(Single-Photon
          Emission Computed
           Tomogmphy,SPECT)等在内的功能成像两类1431。不同模态的成像
           技术有很大互补性,例如MR和CT成像可以提供结构信息,但不能用于病理学诊断。
           SPECT和PET成像可以提供细胞功能和新陈代谢变化信息,却无法定位病变组织的空
           间位置m】。随着科技的进步和社会的发展越来越要求能够早期预测心血管疾病的发生,
           进而制定应对策略。因此,合理有效地利用不同成像模态的优势有利于提高心血管疾
           病诊断的准确性【451。
\subsection{问题的提出与研究意义}
研究表明,年龄、性别和空气质量是心血管疾病的重要诱因【4 ̈91。随年龄增长,
每十年人们患心血管疾病的风险将增加2倒461。据估计,87%死于冠心病的人是60岁
以上的老年人【47】。另有报告指出,男性比经前女性更容易患心血管疾病【481。中年人中,
男性患冠心病的几率甚至达到女性的2倍到5倍f48】。发表于《美国医学会杂志》的一
项研究表明,PM2.5会导致动脉斑块沉积,引发血管炎症和动脉粥样硬化,最终导致
心脏病或其他心血管问题【491。受人口老龄化、性别比例失调、空气污染等因素影响,
中国面临较高的心血管疾病发病风险。中国心血管流行病学多中心联合研究表明,心
血管疾病已经成为中国人民的主要死亡病因,其中缺血性心脏病是最常见的【501。这必
会给有限的医疗资源带来巨大压力。因此,中国迫切需要发展完善心血管疾病早期诊
断和风险评估方法,解决日益增长的心血管疾病给社会带来的负捌引】。
虽然有些诱发心血管疾病的危险因素,如年龄、性别或家族遗传病史,是不可改
变的,但是许多重要的危险诱因可以通过社会变革、改变生活方式或药物治疗得到缓
解【52】。然而,心血管疾病通常是缓慢发生的,早期并无明显症状。患者往往在症状较
严重时才去就医甚至来不及就医就突发死亡。如果能够及时诊断心血管疾病,那么患
者存活机率将大幅提升。因此,借助医学影像技术,定性定量分析心脏的功能参数和
心脏形态变化,对降低心血管疾病死亡率,提高生活质量意义重大。
然而,受成像设备场偏移效应、局部体效应(同一体素中包含多种组织)、患者
检查时体位运动等因素影响,心脏影像常存在噪声、伪影、边缘模糊和信号强度不均
匀(偏移场)等现象。因此,利用偏移场校正和影像噪声消除方法提升心脏影像质量
有助于提高影像处理与分析算法的精度。另一方面,为了成像实时跳动心脏,成像设
备必须能够快速有效地消除呼吸运动和心脏跳动在重建影像上引起的运动伪影。所以,
保证成像质量的同时,尽可能提升心脏成像速度具有重要意义。现代的医学成像技术
给心脏疾病诊断带来4D影像数据的同时,也给传统的依靠临床医生阅读断层胶片诊断
心血管疾病的工作方式带来困难【531。因此,客观地处理这些心脏影像数据,并进行定
量分析对心血管疾病的诊断具有重要的临床意义【541。影像分割是这一类处理方法中最
重要的一个。由分割后的心室影像可以计算心肌质量、射血分数,并能模拟分析整个
心跳过程中心室容量和心肌厚度的变化情况,进而可以准确地定量评价心室功能。类
似地,由分割后的冠状动脉影像可以计算得到动脉截面积、半径/直径的变化情况,进
而可以定位狭窄并估计狭窄程度【551。作为心脏疾病计算机辅助诊断中另一个重要工具,
心脏可视化可以以不同的绘制方式重建并显示3D/4D心脏以及心脏组织的分割结果。
从而可以将放射科医生从断层阅片并想象断层影像间空间关系的模式中解放出来【561。
另外,由于心脏解剖结构复杂且存在较大的个体差异,心脏影像表现为组织结构多且
对比度差等特点。因此,由CT和/或MR影像分割心脏组织,进而分析心脏功能参数,
并可视化实时跳动的心脏及分割结果存在很多难点,具有挑战性【57】。分析相关领域研
究现状,并发现其各自需要面对的问题与挑战,进而找到各难点的解决办法对心脏疾
病诊断意义重大。


\section{国内外研究现状}

通过连续观察一系列心脏影像,结合临床诊断经验由医生给出诊断报告的诊断模
式,使得诊断结果高度依赖于医生的主观经验,给心血管疾病的诊断带来很多不确定
性。特别地,时间分辨率和空间分辨率大幅提高的影像设备在增加影像精度的同时也
增加了临床医生的阅片负担。因此,借助计算机辅助诊断技术,处理分析心脏影像进
而诊断心血管疾病成为目前国内外研究热点【58】。然而,医学成像设备采集到的心脏影
像往往存在灰度不一致性和噪声。为了提高影像处理分析算法的精度,需要对心脏影
像进行偏移场校正和影像噪声消除处理。现有的偏移场校正算法可以归类为前处理方
法和后处理方法。前处理方法的典型代表包括采用放置灰度分布信息己知的物体模型、
多线圈采样、采用特殊的脉冲序列等来降低外界环境对影像灰度影响的成像方法[59-61】。
文献f621指出,后处理的偏移场校正方法主要包括:基于低通滤波模型的方法、超曲面
模型方法、统计模型方法、灰度信息模型方法、基于分割的方法。传统噪声消除方法
通常基于噪声是振荡的,正常组织的灰度是平滑的或分段平滑的假设,将组织微小细
节和噪声等同对待,消除影像噪声的同时也会扭曲影像中的结构信息,给医学影像数
据带来新伪影【631。近些年提出的Non.Local去噪方法有效消除影像噪声的同时,不会
扭曲心脏影像,已经成为当前研究热点【631。经过偏移场校正和噪声消除等影像预处理
后的心脏影像影像有助于提高影像处理分析方法的精度。影像分割是影像处理分析关
键技术之一【641。通常地,分割方法通过影像灰度、形状模型、配准或分类的方法从医
学影像中提取感兴趣区域。由分割结果可以计算并获取心脏全局参数和各种运动参数
以辅助医生进行心血管疾病的早期诊断。心脏可视化是影像处理分析另一关键技术【65】。
该技术可以定性和/或定量建立心脏三维、四维模型,并通过不同的绘制技术将实时跳
动的心脏和分割到的组织器官以不同的颜色渲染出来,且可以显示心脏功能参数信息。


由左心室分割算法定位出心室内膜和外膜后的分割结果可以估计心室体积、心肌
质量和射血比例等左心室功能参数。但是受心室外膜附近糟糕对比度和整个心脏影像
灰度不均匀性的影响,左心室分割仍然是一个开放性的具有挑战的问题【661。手动分割
是最显而易见的左心室分割方法【6‘n。但是,由经过特殊训练的内科医生或技术人员手
动勾勒出左心室分割曲线是一项耗时且乏味的工作。另外,手动分割往往在勾画者内
和勾画者间存在差异。也就是说,勾画者有自己的喜好和倾向。同时,同一勾画者在
不时间对同一影像描绘的分割曲线也不完全相同。除此之外,手动分割一层一层地描
绘左心室内外膜曲线,并没有考虑到心室的空间结构信息。为了克服手动分割的缺点,
领域内研究人员提出了许多自动或半自动的左心室分割方法168-76】。传统的分割技术像
阈值、分类、聚类都被用于左心室分割【6引。活动形状模型(Active
 Shape Model,ASM)
 及由其演化来的活动外观模型(Active
  Appearance
   Models,AAMs)已经成为杰出的左
   心室分割方法[69-71】。但是这类方法严重依赖训练数据集的规模和类别丰富程度。基于
   配准的左心室分割方法需要手动分割数据集做为模版,且算法分割精度受配准精度限
   制【72】。由于可以提供光滑的封闭曲线作为分割结果且可以达到亚像素分割精度,活动
   轮廓模型和水平集方法广泛应用于影像分割领域173-76】。文献[73】提出一种耦合的表面蔓
   延方法来分割左心室心肌。该方法通过控制内膜曲线和外膜曲线将心肌厚度限定在给

定区间。显然,这个限制并不符合心肌的解剖结构。Paragios和Lynch进一步改进了这
种方法[74,75】。他们用两个不同水平集函数的零水平分别表示心室内膜和外膜。两个水
平集函数耦合在一起演化。同样地,两个零水平集之间的距离在迭代过程中被限定在
指定区间。2005年Chung和Vese提出一种多层水平集方法(76】。该方法用同一水平集
函数的多个水平集表达影像对象的多个边界。但是,他们并没有验证这种方法用于心
室分割的有效性,并且该方法并没有在代表影像边界的多个水平集之间增加任何条件
约束。以上提到的方法均不具备处理影像灰度不一致性的能力。2011年发表在Medical
Image
 Analysis上的综述性文章[77】总结了发表于1993到201 1年之间的70篇关于左心
 室分割的文章,并以是否需要先验知识来将左心室分割算法分为两类。医学影像处理
 分析领域顶级会议MICCAI(Medical
  Image Computing
   and Computer—Assisted
   Intervention),于2009年组织了名为“Cardiac MR
    Left
     Ventricle Segmentation
     Challenge”的比赛【78】。该比赛只评价分割算法在收缩末期和舒张末期的分割结果。最
     终出版的会议录收集了8各分割算法,但各算法的分割结果都差强人意。
     
2613年,该
会议又组织了名为“MICCAI
Challenge Workshop on Segmentation:Algorithms,Theory and Applications”的比赛,
其中左心室分割是该workshop的一个重要组成部分【79】。
与 2009年的比赛不同,此次比赛评价整个心跳周期内左心室的分割结果。参赛的9种分
割方法按是否需要配准算法支持被分为两类。组委会公布的排名显示,非配准方法的
分割精度普遍高于基于配准的分割算法。





\section{本文研究内容与创新点}

\section{本文组织结构}
