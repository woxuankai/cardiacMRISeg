% !Mode:: "TeX:UTF-8"

\chapter{绪论}


\section{研究背景与意义}

心血管疾病(Cardiovascular diseases, CVDs)在世界范围仍然是最大的死亡原因之一。
在2008年,超过一千七百万人死于心血管疾病。
令人遗憾的是,在这些死者中,超过三百万人在死亡时不超过六十岁,
而这一切,很大程度上是可以被预防的。
因心血管疾病导致的过早死亡的比例在高收入国家超过4\%,在低收入国家超过42\%,
这导致了在国家和人口之间心血管疾病的发生率的日趋严重不平等。
有新的层面的问题在这个令人震惊的情况发生。
在过去的二十年中,因心血管疾病导致的死亡人数在高收入国家减少,
但在中低收入国家以惊人的速度增加。\citeup{WHOCardiovascularDisease2011}

对于左心室的非入侵式(non-invasive)的评价
是对于心血管疾病的诊断与病症掌握的重要部分。
电影式磁共振成像(cine MRI)已经被证明是一种能定量估计左心室功能的、
准确并且可重现的形态。
相关的估计包括了心室体积,质量和腔体射血分数(ejection fraction, EF),
这些估计都是基于对心内膜(endocardium)与心外膜(epicardium)界限的分割技术的结果。
\citeup{LevelSetMICCAI2013}
%图\ref{CineMRIPresentation}\ref{S1}展示了一次电影式磁共振成像结果的部分结果,
%图\ref{ManualSegOfCineMRI}展示了扫描结果在舒张末期的三维图像的手工分割结果
%\footnote{完整的扫描结果是以三维空间$+$时间(3D$+$T)的形式保存的}
%\footnote{图片来源:MICCAI Challenge Workshop on Segmentation: %
%Algorithms, Theory and Applications ("SATA")}
%\pic[hb]{舒张末期专家分割结果}{width=0.5\textwidth}{ManualSegOfCineMRI}

%\begin{pics}[h]{电影式磁共振扫描结果(部分)\newline}{CineMRIPresentation}
%  \addsubpic{t0-z0}{width=0.14\textwidth}{S1}
%  \addsubpic{t0-z2}{width=0.14\textwidth}{S2}
%  \addsubpic{t0-z4}{width=0.14\textwidth}{S3}
%  \addsubpic{t0-z6}{width=0.14\textwidth}{S4}
%  \addsubpic{t0-z8}{width=0.14\textwidth}{S5}
%  \addsubpic{t0-z10}{width=0.14\textwidth}{S6}\newline
%  \addsubpic{t0-z8}{width=0.14\textwidth}{V1}
%  \addsubpic{t3-z8}{width=0.14\textwidth}{V2}
%  \addsubpic{t6-z8}{width=0.14\textwidth}{V3}
%  \addsubpic{t9-z8}{width=0.14\textwidth}{V4}
%  \addsubpic{t12-z8}{width=0.14\textwidth}{V5}
%  \addsubpic{t15-z8}{width=0.14\textwidth}{V6}
%\end{pics}

快速发展的影像技术带来了海量的影像,也给影像科医生带来了巨大的工作量。
对于电影式磁共振成像得到的的四维影像(三维空间$+$时间)的全手动分割是不现实的,
因为这将浪费一名经验丰富的医师大概两个小时来分割一个典型的有150个切片的数据集。
因此,一般只将完整数据集中的舒张末期和收缩末期的图像进行手动分割。
如果想利用所有得到的心脏磁共振图像来进行量化分析,我们需要全自动的分割方法。
除此之外,手动分割还有一个问题,分割结果是受主观影响的:
同一名医师的不同次分割结果、不同医师的分割结果,都存在差异。
自动化的分割让不同时间、不同地点的分割结果更容易比较,
因为自动话的分割结果是完全定义好的、确定性的过程。
最后,对于一些比较好的方法,无论扫描的方向如何,自动化的分割结果都是一致的,
这对于手动分割来说,是个非常乏味的过程。\citeup{KedenburgAutomatic2006}


\section{多图谱分割方法简介}
多图谱分割方法(Multi-atlas segmentation, MAS)是由Ro, first introduced and popularized by the pioneering work of Rohlfing, et al.
(2004), Klein, et al. (2005), and Heckemann, et al. (2006), is becoming one of the most widely-used and
successful image segmentation techniques in biomedical applications. By manipulating and utilizing the en-
tire dataset of “atlases” (training images that have been previously labeled, e.g., manually by an expert),
rather than some model-based average representation, MAS has the flexibility to better capture anatomi-
cal variation, thus offering superior segmentation accuracy. This benefit, however, typically comes at a high
computational cost. Recent advancements in computer hardware and image processing software have been
instrumental in addressing this challenge and facilitated the wide adoption of MAS. Today, MAS has come a
long way and the approach includes a wide array of sophisticated algorithms that employ ideas from machine
learning, probabilistic modeling, optimization, and computer vision, among other fields. This paper presents
a survey of published MAS algorithms and studies that have applied these methods to various biomedical
problems. In writing this survey, we have three distinct aims. Our primary goal is to document how MAS was
originally conceived, later evolved, and now relates to alternative methods. Second, this paper is intended to
be a detailed reference of past research activity in MAS, which now spans over a decade (2003–2014) and en-
tails novel methodological developments and application-specific solutions. Finally, our goal is to also present
a perspective on the future of MAS, which, we believe, will be one of the dominant approaches in biomedical
image segmentation.\citeup{MultiAtlasSurvy}



\section{国内外研究现状}

尽管非常重要,右心室的全自动分割仍然是一个尚未解决的并且具有挑战性的问题。
\citeup{LevelSetMICCAI2013}
一方面,在电影式磁共振成像中,由于对成像时间有要求,心脏长轴方向分辨率较低,
成像结果的灰度严重不均,心脏的运动也对成像结果有一定干扰;
另一方面,心脏的结构复杂且心内膜周围的组织对比度并不高,难以区分,
而且心脏在一个心动周期内的形变非常大。
这些因素导致许多方法很难达到好的效果。

为了解决这一问题,研究人员提出了许多方法,包括

医学图像处理领域的顶级会议,医学图像计算与计算机辅助干预协会
(The Medical Image Computing and Computer Assisted Intervention Society,
The MICCAI Sociery)分别在2009和2013年组织了名为
``Cardiac MR Left Ventricle Segmentation Challenge''和
``MICCAI Challenge Workshop on Segmentation:
Algorithms, Theory and Applications ("SATA")''的比赛。

为了克服手动分割的缺点,领域内研究人员提出了许多自动或半自动的左心室分割方法168-76】。传统的分割技术像
阈值、分类、聚类都被用于左心室分割【6引。活动形状模型(Active
Shape Model,ASM)
及由其演化来的活动外观模型(Active
Appearance
Models,AAMs)已经成为杰出的左
心室分割方法[69-71】。
但是这类方法严重依赖训练数据集的规模和类别丰富程度。基于
配准的左心室分割方法需要手动分割数据集做为模版,且算法分割精度受配准精度限
制【72】。由于可以提供光滑的封闭曲线作为分割结果且可以达到亚像素分割精度,活动
 轮廓模型和水平集方法广泛应用于影像分割领域173-76】。文献[73】提出一种耦合的表面蔓
延方法来分割左心室心肌。该方法通过控制内膜曲线和外膜曲线将心肌厚度限定在给
定区间。显然,这个限制并不符合心肌的解剖结构。Paragios和Lynch进一步改进了这
种方法[74,75】。他们用两个不同水平集函数的零水平分别表示心室内膜和外膜。两个水
平集函数耦合在一起演化。同样地,两个零水平集之间的距离在迭代过程中被限定在
指定区间。2005年Chung和Vese提出一种多层水平集方法(76】。该方法用同一水平集
函数的多个水平集表达影像对象的多个边界。但是,他们并没有验证这种方法用于心
室分割的有效性,并且该方法并没有在代表影像边界的多个水平集之间增加任何条件
约束。以上提到的方法均不具备处理影像灰度不一致性的能力。2011年发表在Medical
Image  Analysis上的综述性文章[77】总结了发表于1993到201 1年之间的70篇关于左心
室分割的文章,并以是否需要先验知识来将左心室分割算法分为两类。



\section{本文研究内容与创新点}
本文将多谱图分割方法


\section{本文组织结构}
本文共四章。

第一章介绍了问题的提出与该课题的意义。

第二章首先概述了传统的分割方法,
然后从配准、图谱选择、标签融合等方面着重介绍了多图谱分割方法
并简述了分割结果的评价方法。

第三章从多方面描述了本人所做的工作,并对结果做了评价与分析。

最后一章对本文提出的方法做了归纳,并总结了该方法的意义、改进方向与展望。

