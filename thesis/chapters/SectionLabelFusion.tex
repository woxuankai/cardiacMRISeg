\section{图像融合方法} %copy of 基于多atlas的心脏右心室精准分割
在完成图谱选择并进行标签传播之后,我们将得到多个分割结果;
为了得到最终的分割结果,我们需要将多个结果融合起来。

\subsection{多数投票法}
%多数投票法(majority voting)是一种简单但不失有效的方法。
%多数投票法中,对于每一个部位,最终标记结果是所有分割结果中出现次数最多的标记,
%这个方法使用了所有分割结果的信息。
%然而他也有缺点,多数投票法并没有利用图像灰度信息。
%
%加权投票法(weighted voting)是多数投票法的一个扩展,每个图谱都有一个权重,
%权重与图谱和目标图像的相似度有关。
%
%加权投票法如下:
%
%\begin{equation}
%  E_{WV}=max[f_1(x),\ldots,f_i(x)], i=1,2,3,\cdots,L
%\end{equation}
%\begin{equation}
%  f_i(x)=\sum_{k=1}^{K=1}w_{k,i}(x), i=1,2,3,\cdots,L
%\end{equation}
%\begin{equation}
%  w_{k,i}(x)=
%  \begin{cases}
%    1 & i=e_k(x)\\
%    0 & i\ne e_k(x)
%  \end{cases}
%  ,e_k(x)=i,k=1,2,3,\cdots,K
%\end{equation}
%式子中,$L$代表分类器的个数,$w_{k,i}(x)$是加权系数。
%该方法中
多数表决法(Majority voting)是最简单最直接的融合算法,
它接照少数服从多数的方法对标号图像进行融合。
设$S(x)$为待分割图像中像素$x$处的分割标号:
\begin{equation}
  S(x)^\ast=arg\ \underset{c}{max}\sum_{i=i}^nf(L_i^\prime(x),c)
\end{equation}
\begin{equation}
  f(L_i^\prime(x),c)=
  \begin{cases}
    1& L_i^\prime(x)=c\\
    0& L_i^\prime(x)\ne c
  \end{cases}
\end{equation}


\subsection{STAPLE}

\subsection{Jonit Label Fusion}
