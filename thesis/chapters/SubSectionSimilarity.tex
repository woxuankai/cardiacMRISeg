\subsection{相似性测度}\label{SectionMetric}
相似性测度定量化地衡量了两幅图像匹配的效果,是图像配准过程中十分重要的一部分。
一般情况下待配准的图像是在不同时间、不同条件、甚至不同成像技术下获取的,
图像描述的信息可能存在本质的差别,这种情况下就没有绝对的配准问题,
那么我们的任务就是寻找一种准则,使两幅图像在这种准则下达到最佳匹配效果,
即认为在相似性测度取得全局极值的位置两幅图像达到最佳匹配效果。
这里的准则称之为相似性测度,在一些非刚性配准中还加上形变约束。
准则的选择和配准的目的、具体的图像形态、几何变换类型有关。例
如,有些准则允许很大的几何变换搜索范围,而有些准则要求初始位置和最优
的配准结果比较接近才能得到正确的结果;有些准则仅仅适用于同一模态图像
间的配准,‘而有些准则能处理不同模态图像的配准。遗憾的是现在还没有一个
明确的准则来指导在各种情况下如何选择配准的相似性度量准则,更不存在各
种情况下都通用的相似度准则。
既然配准只在某种准则下取得相对最优,准则的选择直接影响着配准的效
果。因此,如何选择合适的相似性度量准则就成为图像配准中一个十分关键的
研究问题,大量的研究论文也表明了这一点。从发表的论文来看,主要有两种
相似性度量准N-基于特征(feature-based)和基于体素(voxel-based)。基于特
征的准则一般是最小化两图像相应特征间的距离,常用的特征有对应解剖结构
中的控制点、二维边缘线、三维表面等。这种准则下,通常先要提取特征,利
用这些局部的特征信息进行配准,特征提取的准确性直接影响着配准的精度;
基于体素的方法是目前的研究热点。从总体上来看,这类准则中常见的有:相
关性测度,包括相关系数、傅立叶域的互相关和相位相关等;总体平均差最小
化;灰度比的方差最小化;互信息最大化。其中基于互信息的方法获得了很大
的成功f27l,大量文献表明,该方法不仅适用于单模态图像配准,对多模态图像
配准问题也取得了不错的结果。
