\subsection{相似性测度}\label{SectionMetric}
%copy of 基于多atlas的心脏右心室精准分割
一般情况下,待配准的图像是在不同时间、不同条件下、甚至是不同成像技术下获取的,
图像描述的信息可能存在本质上的差别,
因此没有绝对的配准问题,实际操作中通常利用相似性度量来度量配准的程度。
评估图像配准算法的优劣基本要考虑算法的
有效性、鲁棒性、复杂性、准确度、高效性和在临床上的一些可行性等。

相似性测度的选择直接影响着图像配准的结果,
相似性度量和特征空间、搜索空间密切相关,
不同的特征空间往往对应不同的相似性度量;
而相似性度量的值将直接决定配准变换的选择,
以及判断在当前所取的变换模型下图像是否被正确匹配了。
通常配准算法抗干扰的能力是由特征提取和相似性度量共同决定的。
因此,如何选择一个合适的相似性测度,
使得它可以准确描述图像之间的相似程度是一个重要的研究方向。
相似性测度的选择和配准的目的、具体的图像形态、
几何变换关系以及特征空间的选择都有关系,
如有些测度仅仅适用于同一模态图像间的配准,
有些测度能处理不同模态之间的相关程度,
有些适用于基于特征点的配准,而有些适用于基于像素强度的配准。
总体来说,相似性测度的选择要综合以上多方面因素才能达到最佳的配准效果。

常用的相似性测度有灰度平均差(Mean Squares, MS)、
归一化相关系数(Normalized Correlation Coefficient, NCC)、
互信息(Mutual Information)、归一化互信(Normalized Mutual Information)、
灰度均方差(Sum of Squared Differences, SSD)、
灰度差绝对值(Sum of Absolute Difference, SAD)等。


\subsubsection{灰度平均差}

\subsubsection{归一化相关系数}

\subsubsection{互信息}

\subsubsection{归一化互信息}

\subsubsection{灰度均方差}

\subsubsection{灰度差绝对值}

