\subsection{相似性测度}\label{SectionMetric}
%copy of 基于多atlas的心脏右心室精准分割
一般情况下,待配准的图像是在不同时间、不同条件下、甚至是不同成像技术下获取的,
图像描述的信息可能存在本质上的差别,
因此没有绝对的配准问题,实际操作中通常利用相似性度量来度量配准的程度。
评估图像配准算法的优劣基本要考虑算法的
有效性、鲁棒性、复杂性、准确度、高效性和在临床上的一些可行性等。

相似性测度的选择直接影响着图像配准的结果,
相似性度量和特征空间、搜索空间密切相关,
不同的特征空间往往对应不同的相似性度量;
而相似性度量的值将直接决定配准变换的选择,
以及判断在当前所取的变换模型下图像是否被正确匹配了。
通常配准算法抗干扰的能力是由特征提取和相似性度量共同决定的。
因此,如何选择一个合适的相似性测度,
使得它可以准确描述图像之间的相似程度是一个重要的研究方向。
相似性测度的选择和配准的目的、具体的图像形态、
几何变换关系以及特征空间的选择都有关系,
如有些测度仅仅适用于同一模态图像间的配准,
有些测度能处理不同模态之间的相关程度,
有些适用于基于特征点的配准,而有些适用于基于像素强度的配准。
总体来说,相似性测度的选择要综合以上多方面因素才能达到最佳的配准效果。

常用的相似性测度有灰度平均差(Mean Squares, MS)、
归一化相关系数(Normalized Correlation Coefficient, NCC)、
互信息(Mutual Information, MI)、
归一化互信(Normalized Mutual Information, NMI)、
灰度均方差(Sum of Squared Differences, SSD)、
灰度差绝对值(Sum of Absolute Difference, SAD)等。

\subsubsection{灰度平均差}
图像$A$,$B$在给定区域内的灰度平均差定义为:
\begin{equation}
  MS(A,B)=\frac{1}{N}\sum^{N}_{i=1}(A_i-B_i)^2
\end{equation}
其中,$N$表示给定区域内的像素个数,
$A_i$和$B_i$分别代表第$i$个像素位置处的灰度值。
该测度在理想情况下最优值为0,表示两幅图像在给定区域的灰度值完全相同。
应用该相似性测度基于如下的假设:两幅图像对应像素点灰度值相同,
因此该准则只适用于同模态图像配准。
该准则计算简单,相对来说可以在一个比较大的范围内搜索匹配,
但对图像灰度值的线性变化比较敏感。

\subsubsection{归一化相关系数}
对待配准图像$A$、$B$在所需评估区域内,其归一化相关系数定义为:
\begin{equation}
  NCC(A,B)=\frac{\sum_i^N(A_{i=1}\times B_{i=1})}%
  {\sqrt{\sum_{i=1}^NA^2_i\times \sum_{i=1}^NB_i^2}}
\end{equation}
该值在理想情况下最优值为1,表示两幅图像之间像素强度值完全相同。
该测度也仅适用于单模态图像的配准算法评估。

\subsubsection{互信息量}
在医学图像配准中,
待配准的两幅图像可能来自于不同的时间或不同的成像设备,
但它们都基于共同的人体解剖位置,
因此我们把两幅图像看成是两个随机变量时,
两者应该具有某种程度的相关性。
给定$A$,$B$两幅图像,两幅图像之间的互信息定义为:
\begin{equation}
  MI(A,B)=H(A)+H(B)-H(A,B)
\end{equation}
\begin{equation}
  H(A)=-\sum_ap_A(a)log\ p_A(a)
\end{equation}
\begin{equation}
  H(A,B)=-\sum_{a,b}p_{A,B}(a,b)log\ p_{A,B}(a,b)
\end{equation}
其中$H(A)$表示$A$的熵,$H(A,B)$是图像$A$和$B$的联合熵,
$p_A(a)$表示像素$a$在图像$A$中出现的概率,
$p_{A,B}(a,b)$表示图像$A$和$B$的联合概率密度。

通过以上几式可得:
\begin{equation}
  MI(A,B)=-\sum_{a,b}p_{A,B}(a,b)log\frac{p_{A,B}(a,b)}{p_A(a)p_B(b)}
\end{equation}

当两幅图像在空间中位置达到一致时,其互信息应该为最大,反之最小。
互信息量不仅可以用于单模态图像配准中,
而且可以用在多模态的图像配准,
可以较好的应用于图像配准的各个领域。

\subsubsection{归一化互信息量}
在互信息量的基础上,Studholume提出了改进的互信息,
即归一化互信息量和熵关系数(Entropy Correlation Coefficient, ECC),
定义如下:
\begin{equation}
  NMI(A,B)=\frac{H(A)+H(B)}{H(A,B)}
\end{equation}
\begin{equation}
  ECC(A,B)=\frac{2I(A,B)}{H(A)+H(B)}
\end{equation}

归一化互信息对于图像重叠的敏感度降低,能够将配准精度提高。

\subsubsection{灰度均方差}
灰度均方差代表两幅图像中所有灰度差的平方和,该值越小,表明图像的相似度越高。
定义如下:
\begin{equation}
  SSD(A,B)=\frac{1}{N}\sum_{\Omega}(a-b)^2
\end{equation}
其中,$\Omega$代表图像的全域,
$a$、$b$分别是图像$A$、$B$中相同位置上像素点的灰度值。
该测读也只适用与同模态的图像配准。

\subsubsection{灰度差绝对值}
灰度差绝对值与灰度均方差类似,它是将图像间灰度差的绝对值进行求和,
所以灰度绝对值使用范围与灰度均方差一样,不能应用在异模态的配准中。
灰度差绝对值定义为:
\begin{equation}
  SAD(A,B)=\frac{1}{N}\sum_\Omega\lvert a-b\rvert
\end{equation}


