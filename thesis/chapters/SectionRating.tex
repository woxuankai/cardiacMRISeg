\section{分割结果评价}%copy 基于多图谱的海马提自动分割方法研究
对分割结果进行评价,可有效地促进现有算法的提高与改进。
评价方法可分为主观评价与客观评价,
主观评价是指以人的主观判断作为评价的标准,
由人的视觉决定分割算法的优良程度,
这样易出现评价结果因人而异、评价不一的状况,
无法满足分割算法及分割结果定性、定量的分析要求。
因此,客观评价方法方为研究的重点,
本文的分割结果亦采用客观评价方法进行分析评价。
客观评价方法又可以分为分析法与实验法,
分析法是指直接地对采用的算法进行评价,
分析其原理、性质与特点,总结该算法的优缺点,
这种评价方法仅考虑算法本身,评价结果也只与算法的过程有关。
实验法则是将算法应用到实际中,对分割结果进行测试与对比,
从而归纳出该算法的性能与特点。
一般而言,在具体实施时,实验法除了需获得实际的分割结果外,
通常还需要参考图像,即医学专家手工分割结果,
并采用一定的评价测度,对它们进行对比计算。
常用的客观评价指标有平均距离、均方差、Dice相似性测度及绝对容积误差等。
\subsection{平均距离}
平均距离$D_{ave}$定义如下:
\begin{equation}
  D_{ave}=\sqrt{\frac{1}{N}\sum_x\frac{1}{m}\sum_{i=1}^m\lVert x-R_i(x)\rVert^2}
\end{equation}
其中$R_i(x)$表示图像$I$到集合$R_s$中,第$i$个元素的变换;
$N$表示像素的个数,$m$表示集合$R_s$元素的个数。
\subsection{均方差}
均方差$V_{ave}$的计算公式为:
\begin{equation}
  V_{ave}=\sqrt{\frac{1}{N}\sum_x\lVert x-D(x)\rVert^2}
\end{equation}
式中的$D$表示变形场,$N$表示像素的个数。
均方差是衡量变形场下形状变量信息的一个标准。
\subsection{Dice相似度}
Dice相似性测度是对分割结果与人工分割的参考图像的重叠率进行评价的,
计算公式如下所示:
\begin{equation}
  Dice(Se,Re)=\frac{2V(Se\cap Re)}{V(Se)+V(Re)}
\end{equation}
其中$Se$表示分割结果,$Re$表示人工分割的参考图像,
$V$表示分割区域的体积大小。
相似性测度Dice值越大,
说明分割结果与人工分割的参考图像的重叠率越高,结果越精确。
\subsection{绝对容积误差}
绝对容积误差是对分割结果与参考图像的体积差进行评价,计算方法如下所示:
\begin{equation}
  E(Se,Re)=\frac{\lvert V(Se)-V(Re)\rvert}{V(Re)}\times 100\%
\end{equation}
其中$V(Se)$表示分割结果的体积大小,$V(Re)$表示参考图像的体积大小。
