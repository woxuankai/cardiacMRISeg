% !Mode:: "TeX:UTF-8"

\chapter{结论}
\section{工作总结}
在这篇文章中,我们系统展示了多谱图分割方法所用到的基础理论,
并且结合理论,比较各种方法优劣,实现了多图谱方法对左心室的分割。
此外,我们还对多谱图分割方法作出了一些的改动,
在以微小准确度的减少为代价的同时大大减少了图像分割的计算量与时间。

\section{工作展望}
本文中,我们采用多图谱方法对左心室分割取得了一定的成果,
但是还有一些问题值得进一步研究。
在未来的工作中,我们将主要探索以下方面:
\begin{enumerate}
  \item{提高配准准确率与鲁棒性}
    本文中,我们最终的准确率Dice只有0.8左右,
    然而在已发表的作品中,准确率可以达到0.9左右;
    以及,从图\ref{MASResultSummary},我们可以看出,
    分割结果的鲁棒性也不是特别好(注意图中的异常值)。
    我认为,在图像的配准部分,还有很多工作可以做。
    例如,为了片面追求效率,我们配准过程中使用的是随机梯度下降法,
    相对于梯度下降法,这种方法收敛速度快,但是很容易陷入局部最小值中。
    在未来的工作里,我们可以考虑使用批量下降法等等替代。
    另外,最终标签融合的过程中,如果计算时间允许的话,
    可以考虑融合更多标签来减小最终结果的不确定性,提高鲁棒性。
    以及,\cite{HigherResolution}中有提到,更高分辨率的图像会有更好的效果。
    随着成像技术的发展,分辨率的提高是可以预见的。
    我们可以考虑在分辨率更高的图像上尝试我们的方法。
  \item{提高配准速度}
    在这篇文章方法的实现中,我们大量使用了Bash脚本。
    Bash脚本语法简练,适合快速开发原型,但是不够灵活,导致整个方法效率较低。
    为了减少处理时间,我们做了很多妥协,但这严重制约了最终结果的准确度。
    在未来的工作中,我们可以考虑用更灵活的Python语言来实现算法。
    如果要投入实际应用中,用C++语言实现算法也是必不可少的。
  \item{更高维度的配准}
    在MICCAI 2013提供的数据中,心脏长轴方向分辨率只有10个像素左右,
    对比短轴方向256个像素,是非常低的,而且部分容积效应明显。
    在尝试中,我们发现,三维配准的效果非常不理想,而且配准收敛时间也比较长。
    因此,在本文中我们只对部分切面进行了二维配准。
    这带来了一个问题:
    我们忽视了整个图像中层与层之间的关系,更没有利用帧与帧之间的关系。
    在以后的工作中,我们将继续尝试三维甚至更高维度的配准,
    或者在低维度的配准中利用高维度的信息作指引。
    另外,在\cite{ShiSuperResolution2013}中,作者提出了一个``超分辨率''的方法,
    值得参考。
  \item{自动选取ROI}
    在图像配准之前,我们需要提取ROI,
    这是所有配准方法、许多图像分割方法固有的通病。
    在本文所述方法中,我们对图像ROI的提取是手动完成的,
    这也带来了一定的工作量(一整个Cine MRI数据要标注两个点,耗时半分钟左右)。
    在未来的工作中,我们可以考虑开发一种全自动提取ROI的方法,
    实现正真的全自动分割。
  \item{全心分割}
    这篇文章中,我们只实现了对左心室的分割,
    但实际上,如果对全心分割,多图谱方法并不会多出多少计算量。
    在以后的工作中,我们可以考虑利用多图谱方法这一特性,实现全心分割。
  \item{半自动方法}由于心尖部位特征较少,配准的效果不是特别好,
    可以考虑加入人工的指引,例如手动标记某些特殊的部位(提供landmark)。
    这只会引入微小的人工工作量,并且可以较大提高分割精度。
  \item{结合多种方法}
    多图谱方法有他的有点,但是也有一些固有的缺陷。
    在未来的工作中,我们可以考虑结合使用其他方法,
    例如水平集方法\citeup{LevelSetMICCAI2013}和机器学习等方法。
\end{enumerate}
