\section{常见医学图像分割方法}%copy of 基于多权重概率图谱的脑部图像分割
针对医学图像的分割问题,近几年来,研究人员做了大量的研究工作,
分别以灰度信息、边缘信息、纹理信息、区域信息等为标准
建立起多种分割模型和分割算法。
常见的医学图像分割算法有:阈值分割法、区域生长法、边缘检测法、聚类法、
基于神经网络的方法、基于小波变换的方法以及基于活动轮廓模型的方法等。

\subsection{阈值分割法}
阈值分割法已经研究了40多年,它是一种最简单、最基础的图像分割算法,
它主要依据图像的灰度直方图信息来确定一个适当的灰度值,
即阈值分割法中的阈值,然后利用这个阈值将图像中的像素区分为不同的类,
就此对图像进行分割。
设$(x,y)$为二维图像$I$上的点,$I(x,y)$是图像$I$上点$(x,y)$处的灰度值,
阈值为$T$,则阈值处理后的二值图像$L$定义为:
\begin{equation}
  L(x,y)=
  \begin{cases}
    a& I(x,y)>T\\
    b& I(x,y)\le T
  \end{cases}
\end{equation}
标注为$a$的像素对应于物体,标注为$b$的像素对应于背景。
通常,按惯例,$a=1$(白),$b=0$(黑)。
由上面的分析可以看出,阈值的选择决定了图像分割的最终质量。
常用的阈值选择方法有:双峰法、最小误差法、Otsu法、最大熵阈值法等。

双峰法适用于图像的灰度直方图呈双峰状且有明显的峰谷,
选择谷点的灰度值作为阈值就可以把目标从图像中分割出来。
双峰法简单易实现,
但如果图像的灰度直方图中没有明显的波峰以及虽然有波峰但直方图的波谷过于平坦,
都不能使用双峰法来确定阈值。

最小误差法来源于贝叶斯最小误差分割法。
其分割思想是选取一个阈值,
通过这个阈值将待分割图像分成目标和背景时错分的概率最小。

Otsu法的基本思想是寻找最优域值来使分割后的两类类间方差最大。
Otsu法被认为是最优的阈值选择方法之一。

最大熵阈值法基于信息论中最大熵准则自动选取阈值。
其基本思想是寻找阈值把图像像素划分为两类,
通过划分后得到目标和背景的熵总值最大,
或是使分割前后图像的信息量差异最小来确定最佳阈值。

阈值分割法是一种简单有效、运算效率高、速度快、易于实现的分割方法,
在待分割图像中不同区域之间像素的灰度的对比度比较大时,
容易得到较好的分割结果。
但阈值分割法分割的结果依赖于阈值的选择,
在不同区域或者目标之间灰度差异不明显或有重叠时,
往往会产生错误分割。
并且阈值分割法通常只采用了图像中像素的灰度信息,
没有使用像素的空间信息,因此阈值分割法容易受噪声和图像非均匀性的影响。

\subsection{区域生长法}
区域生长法即是以像素与其邻域像素的相似性作为生长准则,
将像素邻域内与其相似的像素聚集起来。
区域生长法的分割思想是先在每个待提取的区域确定一个像素作为种子点,
然后根据相似性生长准则,
将种子点邻域内与种子点相似的像素合并到种子点所在的区域。
将这些新像素作为新的种子点,重复上面的操作,
直到触发生长终止条件时,生长终止,分割完成。
由上可以看出,区域生长法有三个关键性步骤:
\begin{enumerate}
  \item 种子点的确定,选取的种子点要能代表该 区域的特性,
    如果种子点选择不当,就很容易产生错误分割。
  \item 生长准则的确定,
    不同生长规则所产生的分割结果不同。
  \item 生长终止条件的确定,生长终止条件也会影响算法的准确性与高效性,
    如果生长终止条件设置不当,就会出现将图像分割成过多的区域或分割不足的情况。
\end{enumerate}

区域生长法是一种计算简单、运行速度快的分割方法。
但区域生长法必须人工给出一个种子点作为生长的起点,
如果种子点选取不当,分割结果就会出现错误,
而且每个种子点只能分割出一个区域,
而对于图像中灰度相近但不相邻的多个区域不能一次分割出来;
区域生长法容易受噪声的影响而使区域内有空洞甚至是根本不连续的区域。

\subsection{边缘检测法}
边缘检测法是基于图像不连续性的分割技术,在医学图像中组织与组织,
组织与背景的交界外都会形成边缘,在图像的灰度发生突变的地方也会形成边缘。

边缘检测法通过对目标边缘的检测来实现分割。
边缘检测的基本原理为:图像中边缘处像素灰度发生突变,
而梯度则可以反映出这种变化,梯度的大小反映出像素灰度变化的剧烈程度。
在图像的边缘区域,图像的灰度变化剧烈,
边缘检测法就是利用图像的梯度来反映出这些地方。
图像一阶导数幅值较大的地方可以反映出图像的边缘,
而图像二阶层导数则反映了图像的明暗侧,
因此可以用图像二阶导数零交叉的地方判断图像的边缘。
但二阶导很少直接用于边缘检测,因为作为二阶导数,
它对噪声有令人无法接受的敏感性,其幅度会产生双边缘,且不能检测边缘的方向。
因此,二阶导数常与其他边缘检测技术组合使用。
边缘检测的基本思想是使用如下两个准则之一来找到图像中灰度快速变化的位置
\begin{enumerate}
  \item 寻找灰度的一阶导数的幅值大于某个指定阈值的位置,Sobel算子、
    Prewitt算子、 Roberts算子、 Krisch算子 为经典的一阶导数边缘检测算子。
  \item 寻找灰度的二阶导数有零交叉的位置,
    二阶导数边缘检测算子则有Laplacian算子、Gauss-Laplacian算子等。
\end{enumerate}

边缘检测法简单易实现,但边缘检测法利用图像的梯度来检测边缘,
因此容易受到噪声的影响,而产生孤立或分小段连续的边缘。
并且对灰度变化复杂和细节较丰富的图像进行分割时,
边缘检测容易产生过多的分割区域而对边缘不明显的图像,可能会得到不连续的边界。

\subsection{聚类法}
聚类法的分割理论是在特征空间用聚类算法将像素聚集起来进行分类,
然后映射回原来的图像空间,从而得到分割结果。
K均值算法、模糊C均值算法为常用的聚类算法。

K均值算法是在特征空间中,初始化K个聚类中心,
然后将图像中的每个像素划分到离它最近的聚类中心的那个类,
对图像中所有的像素完成分类后,计算出此时的聚类中心,
重复执行前面的操作直到新旧两类的聚类中心不改变,迭代终止,分类结束。

图像中存在很多不确定性,如形状不确定性、灰度的非均匀性等。
模糊集合理论中的模糊性能很好的描述图像的不确定性。
模糊C均值算法将模糊理论与聚类算法相结合,通过引入``隶属度''的概念,
将像素划分到隶属度高的区域中,从而完成图像分割。

聚类算法的分割理论是在特征空间将像素聚集起来,聚类算法不需要训练样本,
是一种无监督的统计方法。
聚类算法简单有效,但聚类算法只考虑图像中像素自身的灰度信息,
没有包含像素的邻域信息,因此容易受到噪声及图像不均匀性的影响,
而且聚类算法需要初始聚类数目、聚类中心,这对分割结果影响很大。

\subsection{基于神经网络的分割方法}
神经网络通过对人脑学习过程的抽象与模拟,
建立相应的网络模型来实现某些方面的功能。
神经网络模是一个大型并行连续系统。
神经网络由大量的连接节点组成,每个节点都能执行一些基本操作。
基于神经网络的分割方法的分割原理
是通过样本训练出神经网络中节点间的连接和权重从而得出神经网络,
然后用训练得到的神经网络对新输入的图像数据进行分割。

基于神经网络的分割方法引入了图像中像素的空间信息,因此对噪声和非均匀性不敏感。
但网络结构模型的选择是基于神经网络的分割方法要解决的主要问题,
并且基于神经网络的分割方法需要大量的训练样本,
并且巨量的节点连接会造成计算量大、数据量大的问题。

\subsection{基于小波变换的分割方法}
波变换理论(Wavelet Transform)最早由法国工程师Morlet提出,
因其特有的时空多尺度特性,得到非常广泛的应用。
小波变换基于傅里叶变换发展而来,
其基本思想是通过一簇基本小波函数通过不同尺度的平移或伸缩来构成小波函数集,
以该小波函数集来逼近信号。
小波变换是一种时频分析工具,利用其多尺度特性,能有效的从图像中提取信息。
它可以在频域内体现出信号的时域信息,而在时域内也可以体现出信号的频域信息,
并且在时域和频域都具有良好的局部化性质,这是傅里叶变换所不具有的特性。

基于小波变换的分割理论依据是:
二进制小波变换可以检测出二元函数的局部是否有特异性,
以及特异性的大小是多少。
图像的边缘出现在图像局部灰度发生突变的位置,
这些位置对应于小波变换模值极大值处。
利用小波变换进行医学图像分割的基本思想:首先选取特殊性质的小波函数,
利用小波变换函数将图像进行小波变换,然后利用变换的多尺度特性,
检测出待分割图像中的边缘像素点,
然后根据一定的准测将这些边缘像素点连接起来形成轮廓,从而实现对图像的分割。
小波变换的多尺度性表现在:小波变换在小尺度下检测出的边缘细节丰富,
因此边缘定位精度高,但容易受到噪声的影响。
小波变换在大尺度下,可以有效的降低噪声的影响,
检测出边缘较为稳定,但细节不明显。
因此小波变换可以分析非平稳信号(非均匀性及噪声影响的图像)的能力,
这是经典的边缘检测算子都没有的特性。

\subsection{基于活动轮廓模型的分割方法}
原始的活动轮廓模型(active contour  model,ACM),又称snake模型,
由 Kass 在 1987 年提出。
活动轮廓模型是在图像感兴趣区域内初始化一条演化曲线,
同时赋予该曲线能量函数,对能量函数进行最小化,
使演化曲线向着目标轮廓方向逐步变形与运动,最终收敛到目标边界,
从而得到一个平滑、连续、封闭的目标轮廓,即图像分割的结果。
早期的活动轮廓模型采用显式参数的方式表达演化曲线,被称作参数活动轮廓模型。
但参数活动轮廓模型采用显式参数的形式来表达演化曲线,
不能自然地处理区域的拓扑变化问题,降低了该模型的通用性。
水平集方法(level set method)由Osher和Sethian等分别单独提出,
他们用隐式水平集的方式来表达演化曲线,因此被称为水平集方法或几何活动轮廓模型,
水平集方法能够自然地处理曲线的拓扑变化,因而可以检测多个物体的边缘。

