% !Mode:: "TeX:UTF-8"

\chapter{多图谱方法分割心脏}
为了验证上一章所述多谱图配准理论,
本章实现了对心脏左心室心内膜和心外膜之间心肌的分割。

\section{平台与工具}
在算法的实现中,我们基于Linux(Debian sid)操作系统和Bash与Python语言,
使用大脑功能磁共振成像软件库
\citeup{JenkinsonFSL2012,WoolrichBayesian2009,SmithAdvances2004}
(FMRIB Software Library, FSL)
(Functional MRI of the Brain, FMRIB)
对图像进行预处理,
使用ANTs\citeup{ANTsRegistration,ANTsBspline}、
SimpleITK\citeup{LowekampDesignOfSimpleITK2013}和
ITK\citeup{ITKBookDevelop,ITKBookDesign,InsightToolKit}工具
进行配准、图谱选择、标签融合、度量分割结果准确性等操作,
并使用了GNU Parallel\citeup{GNUParallel}工并行加速。
所有实验在一台配置为Intel i5-4200M CPU, 4G$\times$2 1600MHz RAM, 
配有SSD的PC上完成。

在本文所有图的绘制中使用了
SciPy\citeup{SciPy,ScientificComputingPython1,ScientificComputingPython2}
套件里的Matplotlib\citeup{Matplotlib}和NumPy\citeup{NumPy},
使用了ITK-SNAP\citeup{ITKSnap}软件显示磁共振图片以及分割结果。
同时,
Inkscape (\href{http://inkscape.org}{inkscape.org})
和 GIMP (GNU Image Manipulation Program, \href{http://gimp.org}{gimp.org})
软件在绘图过程中也起了很大作用。

%ants、ITK简介?凑字数
\subsection{ITK}
ITK(Insight Segmentation and Registration Toolkit)
是一种跨平台的、开源的应用程序开发框架,
被广泛用于图像分割和图像配准的开发。
图像分割是辨别和分类数字采样得到的数据的一个过程。
例如,从CT或MRI扫描仪等医疗仪器中得到的数据。
图像配准是一个对齐或者是寻找数据之间对应关系的任务。
例如,在医疗环境中,可以将CT扫描与MRI扫描对准,
以便融合包含在两者中的信息。

ITK由美国国家医学图书馆(National Library of Medicine U.S.)资助开发,
作为分析可见人类项目(Visible Human Project)图像的开放算法资源。
ITK提供了非常前沿的、在二维、三位或者更高维度上配准和分割算法。
ITK使用CMake构建环境来管理配置过程。
该软件是用C++实现的,也提供了Python和Java的接口。
这使得开发人员可以使用各种编程语言来创建软件。
ITK的C++实现风格被称为通用编程(即使用模板代码)。
这样的C ++模板意味着代码是高效的,
并且在编译时和运行时运行程序时会发现许多软件问题。
\citeup{InsightToolKit}
实验中,我们使用目前最新的稳定版本ITK4.10。
\subsection{SimpleITK}
SimpleITK是建立在ITK之上的、简化的接口,
它存在的目的是方便ITK在快速原型设计、教育、解释型语言方面的使用。
SimpleITK的主要特点如下:
\citeup{LowekampDesignOfSimpleITK2013}
\begin{enumerate}
\item 提供Python, Java, CSharp, R, Tcl 和 Ruby的C++库。
\item 面向对象设计。
\item 提供简化的、方便使用的、没有模板类的程序接口。
\item 在开源的Apache 2.0许可证下发布。
\item 为Python,Java和CSharp提供二进制发行文件。
\end{enumerate}

\subsection{ANTs}
ANTs(Advanced Normalization Tools)也是基于ITK开发的一款软件,
提供了正则化、图像分割、模板建立等量化形态学分析工具。
与SimpleITK不同,ANTs提供的是能直接运行的程序(工具)而不是库。
许多ANTs配准工具是微分同胚的,但也有形变变换(elastic和BSpline)的。
ANTs还提供了一些独占的组件,例如多变量相似性测度方法,
地标指导(landmark guidance),使用标签图片指导映射,
以及不同形态的贪婪和时空优化实现。

\section{实验数据}

我们使用MICCAI 2013 SATA Workshop on  Cardiac Left Ventricle Segmentation
提供的数据对我们的算法进行检验。
MICCAI 2013数据集共有155个电影式磁共振图像。
其中用于训练的数据83个,用于测试的数据72个。
每个数据最多包含30帧最少包含15帧三维图像。
每个体数据由12到30个切片组成。
每层图像分辨率为$256\times256$。
这些数据同样来自心脏图谱工程
\citeup{FonsecaCAP2011,SuinesiaputraBigHeartData2015}
(Cardiac Atlas Project, CAP)。

由于我们没有测试数据集的金标准,所以只能从训练数据集中挑选。
本文中,我们选取DET0002701,DET0009301,DET0016101,DET0001101,DET0001401,
DET0001701,DET0002501号数据作为实验数据(d0-d6)。
在实验中,我们分别将 DET0002701 (d0) 和 DET0009301 (d1) 作为待配准数据,
使用其它数据对其进行分割;
分割完成后,以数据提供的分割结果作为标准分割(金标准),
将多图谱方法得到的分割结果与标准分割进行比较。

值得注意的是,每一个数据都是四维的。例如,
DET0002701号数据(d0)共20帧,每帧数据$x,y,z$为$256,256,12$,
$z$为心脏长轴方向。

图\ref{OneVolume}展示了DET0002701号数据的第1帧图像,
每一个分图都是垂直于长轴方向一个切片。
\pic[htbp]{DET0002701号数据第1帧图像(从左到右、从上到下为心尖到心脏基部)}%
{width=\textwidth}{OneVolume}

从图中我们可以看到,在一帧图像中,心脏尖部分特征较少,
而心脏基部结构复杂,包含了肺动脉、主动脉等等结构,
在实践中,我们也发现心脏底部和心尖难以配准。
另外,长轴方向的分辨率较低(10像素左右),所以三维图像配准效果不好。
因此,在本文的实验中,我们只选取心脏长轴方向中间的切片进行二维配准。

图\ref{The7thSlice}展示了DET0002701号图像的第9层数据在一个心动周期内的变化。
\pic[htbp]{DET0002701号数据第9层图像(从上到下、从左到右为一个心动周期)}%
{width=\textwidth}{The7thSlice}

观察可以发现,一个心动周期之内,左心室壁的厚度变化和右心室的形变非常明显,
而且心脏内膜附近还有乳头肌等一系列组织。


\section{图像配准}
\subsection{ROI选择}
在图像配准之前,我们首先需要设置我们感兴趣的区域(Region of Interest, ROI)。
设置ROI的目的,是为了是尽量使需要配准的组织占整个ROI足够大的区域;
若待配准组织在ROI占比过小,配准效果可能不理想。
ROI的设置与配准结果直接相关,选择的原则如下:
\begin{enumerate}
  \item ROI必须包含心脏的所有组织。
  \item 在包含所有心脏组织的前提下,ROI必须尽可能的小。
\end{enumerate}

本文中,我们采取手工选择ROI的方法,将所有样本进行提取。
图\ref{ROIAfter}展示了对\ref{ROIBefore}手工选择ROI的结果。
\begin{pics}[htbp]{选取ROI示例}{ROI}
  \addsubpic{原始切面}{width=0.3\textwidth}{ROIBefore}
  \addsubpic{手工选取的ROI}{width=0.3\textwidth}{ROIAfter}
\end{pics}

\subsection{灰度校正}
由于磁共振采集到图像容易产生灰度不均以及某些部位灰度异常的情况,
我们需要在配准之前对图像灰度进行校正。
又由于多图谱方法中,最后还会进行图谱挑选、融合分割结果等步骤,
所以我们采取了非常简单的灰度校正方法:
统计像素灰度值并排序,
然后将第5\%和95\%的像素的灰度值设置为最大值和最小值后重新分配灰度值,
得到的图像再进行配准。这个方法与图像加窗显示类似。

图\ref{WinBefore}\ref{WinAfter}分别展示了灰度校正之前和之后的结果,
图\ref{WinBefore}中,图像胸腔前侧高亮部分导致整个图像看起来较暗。
\begin{pics}[htbp]{图像相似度测度方法比较}{WinComparation}
  \addsubpic{原图}{width=0.3\textwidth}{WinBefore}
  \addsubpic{校正后的结果}{width=0.3\textwidth}{WinAfter}
\end{pics}

\subsection{配准策略}
\subsubsection{多分辨率配准}
在图像配准时,采用多分辨率配准方法有利于改进配准的速度、精确度和自动化程度。
在图像像素较少的地方采用粗糙的比例,产生一个映射,
该映射被用来在接下来较好的图像中初始化配准,
重复这个过程直到达到最好的比例范围,是一种分步计算逐步推进的方法。
具体配准实验时,对较低分辨率图像进行搜索可采用较大步长,
这样将配准参数控制在最优解附近,然后作为初始值在更高一级分辨率进行下一步搜索,
由于搜索空间的缩小能加快优化过程,因而达到快速配准进而加速整个配准过程。

在ITK中,多分辨率配准要求浮动图像、参考图像在同一类型格式上,
并通过多分辨率金字塔滤波器将其转换成等层次的``金字塔'',
采用递归方式对其进行操作,开始采用粗糙的配准,
以后逐步采用较细的配准,重复操作直到达到最好水平位置。
如图\ref{MultiResolution}所示。
\pic[htbp]{多分辨率配准策略\citeup{ITKBookDevelop}}%
{width=0.8\textwidth}{MultiResolution}

在本实验中,我们选择四层分辨率配准,
分辨率分别为原图分辨率的$\frac{1}{8}$,$\frac{1}{4}$,$\frac{1}{2}$和$1$。
在对原图下采样完成后,我们还对采样后的图像进行模糊处理,
在每一层次图像,高斯模糊中参数$\sigma$分别为3,2,1和0。


\subsubsection{多步骤配准}
由于心脏运动是非刚性的,所以毫无疑问的,我们在配准中需要使用非刚性变换。
然而,如果在非刚性配准之前,图像有较大的位置差异,
不仅仅配准时间较长,配准结果也很不好。
所以,我们在进行非刚性变换配准之前,先进行粗配准,
尽量减少图像之间的差异。

因此,我们采用以下配准策略:
刚性变换配准---仿射变换配准---非刚性配准,多步骤地进行配准。

图\ref{RegistrationAffineOnly}展示了将只使用仿射变换配准后得到的变换
应用于图谱图像的结果。
图\ref{RegistrationWarpedAtlasImage}展示了将使用非刚性形变模型配准得到的变换
应用与图谱图像的结果。
\begin{pics}[htbp]{多步骤配准示例}{RegistrationMultiProcess}
  \addsubpic{图谱图像}{width=0.23\textwidth}{RegistrationAtlasImage}
  \addsubpic{只使用仿射变换}{width=0.23\textwidth}{RegistrationAffineOnly}
  \addsubpic{使用非刚性形变模型}{width=0.23\textwidth}%
    {RegistrationWarpedAtlasImage}
  \addsubpic{目标图像}{width=0.23\textwidth}{RegistrationTarget}
\end{pics}



\subsection{配准结果}
图\ref{RegistrationFramework}所示配准框架,
在本文中的实现如图\ref{RegistrationFrameworkPar}所示。

\pic[htbp]{本文中所使用配准方法}{width=\textwidth}{RegistrationFrameworkPar}

本实验中,非刚性配准使用的BSpline形变模型,
优化器使用计算量较少的随机梯度下降法。
以及,处于对磁共振图像灰度不一致性的考虑,
我们在所有的配准步骤中都使用互信息量作为相似度量化测度。

使用上述方法,示例配准结果如图\ref{RegistrationResult}。

\pic[htbp]{配准结果示例}{width=\textwidth}{RegistrationResult}

图中,我们将图谱图像与目标图像配准后,将得到的变换作用于图谱图像,
以展示配准效果。
可以看到,变换后的图谱图像与目标图像已经比较相似了,
我们可以认为配准是比较成功的。

另外,我们还将得到的变换作用于一个网格图像,
得到的结果用来展示配准得到的图像变换。




\section{标签传播}
完成配准之后,我们将配准得到的图像变换应用到图谱标签上,
得到的便是对目标图像分割的一个估计。
在本实验中,不同标签是用不同像素值来区分的,
高阶次的插值可能产生不同的像素值,故我们只能使用最邻近插值。
具体如图\ref{LabelPropagation}所示。
图中,我们将配准得到的变换同时施加于图谱图像和标签,
并将变形后的图谱图像和标签与目标图像和标准分割尽心比较。
通过比较,我们的可以认为,我们的图谱分割方法是行之有效的。
\pic[htbp]{标签传播示例}{width=\textwidth}{LabelPropagation}



\section{图谱选择}
如上一章中所述,在图谱配准完成后,
我们需要选择一些合适的图谱来生成最后的分割结果。

在本实验中,我们尝试了使用互信息量、互相关和灰度均方差来进行挑选。
图\ref{MetricMI},\ref{MetricCC}和\ref{MetricMS}展示了各种方法的效果。
图中,纵坐标是变形后图谱的标记与目标图像的分割金标准的 Dice 值,
横坐标是变形后图谱的图像与目标图像分别做互信息、互相关和灰度均方差。
\begin{pics}[htbp]{图像相似度测度方法比较}{MetricComparation}
  \addsubpic{互信息量}{width=0.45\textwidth}{MetricMI}
  \addsubpic{互相关}{width=0.45\textwidth}{MetricCC}
  \addsubpic{灰度均方差}{width=0.45\textwidth}{MetricMS}
\end{pics}

从图中可以很明显的看到,使用互信息量作为相似度的评价标准,
明显优于其它两种方法。

因此,选用互信息量作为图谱选择的依据。

除了利用图像的灰度信息,我们还从数据本身出发,挑选合适的图谱。

从图\ref{OneVolume}中我们可以看到,
心脏长轴切片中,不同位置的形态差别是非常大的,
因此,我们可以考虑从长轴切片的位置选择合适的图谱。
选择方法也十分简单,虽然每个电影式磁共振图像的长轴方向切片个数不尽相同,
但是第一张都是心尖,最后一张都是心脏基部,所以可以计算出大概位置。


\section{二次配准}
由于非刚性配准的计算量非常大,如果图谱数量比较多,将消耗大量的时间。
于此同时,刚性变换和仿射变换配准则非常快。
例如,在本实验中,BSpline方法非刚性配准的的过程中,
刚性变换和仿射变换配准的只消耗不到1秒,
而整个非刚性配准过程需要2分多钟。
所以,我们提出了一个二次配准方法。
整个方法分割流程如图\ref{MyMASFlowchart}。
\pic[htbp]{分割流程图}{width=0.4\textwidth}{MyMASFlowchart}
\begin{enumerate}
  \item 将所有的图谱和待分割图像使用仿射变换进行配准。
    由于仿射变换下的配准计算量小,
    所以虽然图谱数量众多,但是总的时间并不多。
  \item 使用上文提到的图谱筛选的方法,筛选一部分相似度比较高的图谱。
  \item 将上一步中筛选出的较好的图谱和待分割图像进行非刚性配准。
    虽然非刚性配准计算量较大,
    但是经过上一步的筛选,需要配准的图谱数目大大减少,
    所以总的耗费时间依旧在可接受范围之内。
  \item 将上一步变形后的图谱再一次进行筛选。
    由于之前的筛选是基于仿射变换的,所以筛选的准确度并不是特别高。
    在非刚性配准后进行进一步配准,筛选掉不好的图谱,
    提高图谱融合准确度的同时,也能提高图谱融合的速度。
  \item 将上一步筛选出的图谱中的标签融合,得到最终结果。
\end{enumerate}


\section{标签融合}

标签融合,主要有两个关键词:融合方法、融合数量。

图谱融合方法很多,在本文中,我们尝试了两种融合方法:
一种是非常简单的多数投票法,另一种是STAPLE方法。
图\ref{LabelFusion}分别展示了两种融合方法的效果。
\pic[htbp]{标签融合示例}{width=\textwidth}{LabelFusion}

图中上侧五个分割是图谱方法得到的结果,
第二行左侧是多数投票法得到的结果,右侧是STAPLE方法融合结果,
中间是数据的标准分割(金标准)。
对比之下,不难看出,融合后的标签与融合之前的标签相比,
分割的准确性确实好了很多。

关于融合数目,在保证准确度的情况下,融合的标签数目越少越好;
越多的融合数目意味着更多的非刚性配准与更慢的标签融合过程。
另外,文献\cite{HeckemannMultiClassifier2006}中提到,在合理的猜测下,
标签融合的结果$d$(Dice)和标签的数目$n$有如下关系:
\begin{equation}\label{ndRelationship}
  d(n)=a-\frac{b}{\sqrt{n}}
\end{equation}
式子中的$a,b$是常数,需要通过实验求出,并且满足 $0\le a\le 1$ 与 $b>0$。
该式子表明,随着标签数目的增加,随机误差减少,融合结果的准确度增加,
但是,融合结果的准确度是有上限的。

我们任意选取4个切片,计算融合方法、融合数目和最终结果准且性的关系,
如图\ref{LabelFusionNumAndMethod1}, \ref{LabelFusionNumAndMethod2},
\ref{LabelFusionNumAndMethod3}, \ref{LabelFusionNumAndMethod4}所示
(n只取到10,因为过多的图谱会导致计算时间快速上升,在实际应用中价值不大)。
\begin{pics}[htbp]{结果准确性与融合方法和融合标签数目关系}{NMComparation}
  \addsubpic{d0-v0-s7}{width=0.45\textwidth}{LabelFusionNumAndMethod1}
  \addsubpic{d0-v0-s8}{width=0.45\textwidth}{LabelFusionNumAndMethod2}
  \addsubpic{d1-v2-s4}{width=0.45\textwidth}{LabelFusionNumAndMethod3}
  \addsubpic{d1-v0-s4}{width=0.45\textwidth}{LabelFusionNumAndMethod4}
\end{pics}

从图上我们可以看到,
准确度与融合的标签个数的关系基本符合式\ref{ndRelationship}的,
综合准确度与处理时间,我们令$n=6$。
另外,在上一节提到的仿射变换配准之后,
我们挑选$3n=18$个图谱进行非刚性配准。
在标签融合算法上,从图上我们可以看到,STAPLE方法并没有比多数投票法的效果好太多,
但是运算量却增加了很多,所以我们选择多数投票法。
值得注意的是,虽然分割结果与标准分割相似度较高,
但是相比之下,融合结果的边缘比较粗糙。
提高图谱融合个数可以解决这一问题。


\section{分割结果}
我们将 DET0002701 所有 $z=7$ 的数据(心脏长轴方向切片)和 
DET0009301 中所有 $z=4$ 的数据按照上述方法进行多图谱配准,
结果如下。

图\ref{MASResult}展示了每一个分割结果,
图\ref{MASResultSummary}展示了分割结果统计数据。
\begin{pics}[htbp]{分割结果展示}{NMComparation}
  \addsubpic{单个结果}{width=0.45\textwidth}{MASResult}
  \addsubpic{结果统计}{width=0.45\textwidth}{MASResultSummary}
\end{pics}

\section{本章小结}
本章首先介绍了所用到的工具,然后结合上一章多图谱分割方法的理论基础,
针对电影式磁共振图像与心脏特点,在多图谱方法的框架之下,选择调整参数。
设计预处理、配准、图谱选择、融合等方法,完成了多图谱方法对左心室的分割。
