% !Mode:: "TeX:UTF-8"

\chapter{多图谱方法分割心脏}
为了验证上一章所述多谱图配准理论,
本章实现了对心脏左心室心内膜和心外膜之间心肌的分割。

\section{平台与工具}
在算法的实现中,我们基于Linux(Debian sid)操作系统和Bash与Python语言,
使用ANTs\citeup{ANTsRegistration,ANTsBspline}、
SimpleITK\citeup{LowekampDesignOfSimpleITK2013}和
ITK\citeup{ITKBookDevelop,ITKBookDesign,InsightToolKit}工具
进行配准、图谱选择、标签融合、度量分割结果准确性等操作,
并使用了GNU Parallel\citeup{GNUParallel}工并行加速。

%ants、ITK简介?凑字数
\subsection{ITK}
ITK(Insight Segmentation and Registration Toolkit)
是一种跨平台的、开源的应用程序开发框架,
被广泛用于图像分割和图像配准的开发。
图像分割是辨别和分类数字采样得到的数据的一个过程。
例如,从CT或MRI扫描仪等医疗仪器中得到的数据。
图像配准是一个对齐或者是寻找数据之间对应关系的任务。
例如,在医疗环境中,可以将CT扫描与MRI扫描对准,
以便融合包含在两者中的信息。

ITK由美国国家医学图书馆(National Library of Medicine U.S.)资助开发,
作为分析可见人类项目(Visible Human Project)图像的开放算法资源。
ITK提供了非常前沿的、在二维、三位或者更高维度上配准和分割算法。
ITK使用CMake构建环境来管理配置过程。
该软件是用C++实现的,也提供了Python和Java的接口。
这使得开发人员可以使用各种编程语言来创建软件。
ITK的C++实现风格被称为通用编程(即使用模板代码)。
这样的C ++模板意味着代码是高效的,并且在编译时和运行时运行程序时会发现许多软件问题。
\citeup{InsightToolKit}
\subsection{SimpleITK}
SimpleITK是建立在ITK之上的、简化的接口,
它存在的目的是方便ITK在快速原型设计、教育、解释型语言方面的使用。
SimpleITK的主要特点如下:
\citeup{LowekampDesignOfSimpleITK2013}
\begin{enumerate}
\item 提供Python, Java, CSharp, R, Tcl 和 Ruby的C++库。
\item 面向对象设计。
\item 提供简化的、方便使用的、没有模板类的程序接口。
\item 在开源的Apache 2.0许可证下发布。
\item 为Python,Java和CSharp提供二进制发行文件。
\end{enumerate}

\subsection{ANTs}
ANTs(Advanced Normalization Tools)也是基于ITK开发的一款软件,
提供了正则化、图像分割、模板建立等量化形态学分析工具。
与SimpleITK不同,ANTs提供的是能直接运行的程序(工具)而不是库。
许多ANTs配准工具是微分同胚的,但也有形变变换(elastic和BSpline)的。
ANTs还提供了一些独占的组件,例如多变量相似性测度方法,
地标指导(landmark guidance),使用标签图片指导映射,
以及不同形态的贪婪和时空优化实现。

\section{实验数据}

我们使用MICCAI 2013 SATA Workshop on  Cardiac Left Ventricle Segmentation
提供的数据对我们的算法进行检验。
MICCAI2013数据集共有155个电影式磁共振图像。
其中用于训练的数据83个,用于测试的数据72个。
每个数据最多包含30帧最少包含15帧三维图像。
每个体数据由12到30个切片组成。
每层图像分辨率为$192\times192$。
这些数据同样来自心脏图谱工程
\citeup{FonsecaCAP2011,SuinesiaputraBigHeartData2015}
(Cardiac Atlas Project, CAP)。
由于我们没有测试数据集的金标准,所以只能从训练数据集中挑选。
本文中,我们选取DET0002701号数据(d0)作为目标图像,DET0009301,DET0016101,DET0001101,DET0001401,DET0001701,DET0002501号数据作为图谱(d1-d6)。
值得注意的是,每一个数据都是四维的。
DET0002701号数据(d0)共20帧,每帧数据$x,y,z$为$256,256,12$。
图\ref{OneVolume}展示了DET0002701号数据的第1帧图像,
图\ref{The7thSlice}展示了DET0002701号图像的第9层数据在一个心动周期内的变化,
注意比较左心室壁的厚度变化和右心室的形变。
\pic[htpb]{DET0002701号数据第1帧图像(从上到下、从左到右为心尖到心脏基部)}{width=\textwidth}{OneVolume}
\pic[htpb]{DET0002701号数据第9层图像(从上到下、从左到右为一个心动周期)}{width=\textwidth}{The7thSlice}

\section{图像配准}


\section{图谱选择}

metadata\par
\footnote{在图的绘制中使用了
SciPy\citeup{SciPy,ScientificComputingPython1,ScientificComputingPython2}
套件里的Matplotlib\citeup{Matplotlib}和NumPy\citeup{NumPy}工具。}

\section{图谱融合}

\section{不足与提高}

提高分辨率\citeup{HigherResolution}

结合levelset方法,李纯明老师论文

长轴方向分辨率不够,无法进行3D配准
Cardiac Image Super-Resolution with Global Correspondence Using Multi-Atlas PatchMatch

landmark\par
\section{本章小结}
