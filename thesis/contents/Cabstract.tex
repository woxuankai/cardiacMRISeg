% !Mode:: "TeX:UTF-8"

\begin{Cabstract}{医学图像处理}{多谱图}{配准}{分割}{心脏磁共振}
  心血管疾病对人类健康造成了巨大的威胁。
  研究表明,左心室形态结构与功能异常和心血管疾病
  (如心力衰竭,缺血性心脏病等症)的预测和诊断有直接的关系。
  核磁共振成像以其独特优势在目前心脏疾病临床诊断中颇受欢迎。
  磁共振成像速度与质量的提升带来了海量的数据,
  也给人工图像分割带来了巨大且枯燥的工作量。
  综上,对心脏磁共振图像的全自动分割成为了研究的热点。
  由于存在非刚性形变,形态差异大,组织对比度低,成像效果差等问题,
  心脏难以自动化分割。
  多图谱方法在大脑图像分割领域取得了巨大的成功。
  本文结合心脏图像特点,对多谱图方法稍作改动,
  成功使用多图谱分割方法完成了对左心室的分割。
\end{Cabstract}

