% !Mode:: "TeX:UTF-8"

\chapter*{Multi-atlas based segmentation of brain images:
Atlas selection and its effect on accuracy}
\section*{Abstract}
\section*{Introduction}
\section*{Methods}
\subsection*{Background: multi-atlas segmentation}
\subsection*{Atlas selection}
\subsubsection*{Motivation and background}
The size of the atlas database can affect
various aspects of the process of multi-atlas segmentation
as well as the quality of the final segmentation.
As discussed in Background: multi-atlas segmentation,
the average segmentation accuracy 
achieved by fusing random sets of atlases increases asymptotically
as the number fused becomes large.
This asymptotic increase in accuracy means that
there are diminishing returns in using larger and larger numbers of atlases.
For a large atlas database, however,
the increased computational cost of registering
large numbers of atlases to the query image is an immediate practical problem.
A second difficulty relates to the way
in which an anatomical structure might vary across the population.
If a structure is represented by, say,
two morphologically distinct variants in the population
(and in the atlas database),
then fusing a large number of atlases may give a shape
that does not represent either variant very well.
In such circumstances, for a given query subject,
only a proper subset of the atlases in the database is appropriate to use
— those sharing the variant represented in the query.
Finally, in our experience,
the fusion of a large number of atlases
is more likely to create a smooth estimate of the structure
being segmented and yet a shape which is less smooth may be a better estimate.

For such reasons, the selection of a limited number of atlases,
appropriate for the query subject and prior to multi-atlas segmentation,
would appear preferable to the fusion of
an arbitrarily large number of atlases.
Furthermore, it is natural to ask
whether the asymptotic level of accuracy given by
fusing large numbers of random atlases is the best that can be achieved,
i.e. whether it is possible to equal or exceed this accuracy
by fusing a smaller number of selected atlases.

In previous work on atlas-based segmentation,
the term `selection' has typically been used to describe
the identification of the single best atlas for propagation to a query
(see, for example, Rohlfing et al. (2004a) and Wang et al. (2005)).
Han et al. (2008)
compare the selection of a single atlas against
the propagation and fusion of their entire atlas database.
Our earlier work (Aljabar et al., 2007) and
the work in this paper contrasts with these approaches
as we apply selection to whole sets of atlases prior
to propagation to the target and decision fusion.
Klein et al. (2008) use a similar approach
where multi-atlas segmentation is carried out
using only atlases that reach a user-defined threshold of similarity
with the target.
For a given threshold, the number of atlases used for different targets
may vary.
Our work contrasts this by fixing (for each experiment) the number of atlases
selected for different targets.
This enables an assessment of the effect of selection on segmentation quality
across a range of subjects.

The selection of atlases for segmenting a particular query image
also has parallels with clustering problems. For a large atlas database,
it is possible to search for clusters or modes of the population it
represents. Given an unseen query image and an estimate of the
cluster it belongs to, a better segmentation is expected using images
from the same cluster rather than from different ones. See Blezek and
Miller (2007) and Sabuncu et al. (2008) for examples of clustering
approaches applied to brain MR images.


\subsubsection*{Proposed methods for selection}
Given a large atlas database, a fixed subset size and a query image,
it is theoretically possible to identify the optimal subset of atlases for
generating a multi-atlas segmentation of the query. Leaving aside the
question of how segmentations given by different subsets can be
compared, exhaustively searching all possible subsets is clearly
impractical for a large database.

We therefore present an alternative heuristic approach that uses a
measure of the ‘similarity’ of each atlas to the query subject. Once
assigned, this measure can be used to rank the atlases. Multi-atlas
segmentation can then be carried out using a number of the top-ranked
atlases. A simple approach is to interpret similarity as image similarity,
i.e. to derive it from the intensities in the query and atlas images.
Alternatively, similarity may be derived from meta-information relating
to the subjects. The ranking of each atlas is then based on how closely
the atlas subject matches the query in terms of a clinical variable, such
as age, pathology, clinical history, genetics, gender, handedness, etc.

\subsubsection*{Selection using image similarity}
A selection framework that relies on evaluating the similarity of a
pair of images (an atlas and a query image) requires an estimate of the
correspondence between them. It is possible to align all the atlases to
each new query prior to making a selection. In order to avoid the
computational burden of a large number of registrations direct to the
query, we apply a selection framework that makes use of a standard
space defined by a reference image.

This approach identifies an arbitrary reference image in advance
and all atlases in the database are aligned to it. When a new query
image is given, it is also aligned in the same way to the reference. The
image similarity of the query image and each of the atlases can then be
evaluated, since they are all aligned. These similarity values can then
be used to assign ranks. The top-ranked atlases are selected and
registered directly to the query in order to generate a multi-atlas
segmentation estimate in the native space of the query image. This
approach uses two types of registrations: those that align images to
the reference prior to selection and those used to propagate atlases
when generating the multi-atlas segmentations. This selection framework
is illustrated schematically in \ref{TranslationSelection}. 
\setcounter{chapter}{5}
\pic[htpb]{ Multi-atlas segmentation with image similarity selection.
Left: All the atlas anatomies $A_i$
and the query image $Q$ are registered to the reference image $R$.
Similarities between the spatially normalised query and each of the atlases
are used to generate ranks.
Right: Top-ranked atlases are selected and registered directly
to the query image.
The selected atlas segmentations $L_i$ are propagated to the query
giving the segmentations $L^\prime_i$
which are fused to generate the native space segmentation estimate $L_Q$
for the query image.
}{width=0.5\textwidth}{TranslationSelection}

Image similarity can be expressed using a variety of metrics,
including sums of squared differences (SSD), cross-correlation (CC),
mutual information (MI) (Collignon et al., 1995; Viola and Wells, 1995)
and normalised mutual information (NMI) (Studholme et al., 1999). In
the context of image registration, information theoretic measures such
as MI and NMI are intended for aligning multi-modality images. In the
context of selection, this makes such measures more appropriate for
images with widely differing levels of contrast and appearance, for
example if they were acquired on different MR scanners.

A related choice concerns the region over which to evaluate the
similarity metric. The region of interest (ROI) where the similarity
metric is evaluated can be the complete overlap of the atlas and query
images, or it can be made more specific. For example, if a hippocampal
segmentation is required, the ROI might represent a suitably located
region that is large enough to be likely to encompass the target
hippocampus and yet small enough to avoid evaluating the similarity
metric over regions distant from the structure of interest that have
little effect on its segmentation.

Another choice relating to image similarity selection concerns the
transformations used during the registrations. These can be rigid,
affine or non-rigid and, in the non-rigid case, the degree of local
control (or flexibility) can be varied. As mentioned above, the spatially
normalising transformations that are carried out prior to selection
(\ref{TranslationSelection} left)
need to be distinguished from the transformations used to
propagate atlas segmentations after selection and directly to the
query image (\ref{TranslationSelection} right).
The principle we have adopted for spatially
normalising transformations is that they should correct for gross
differences in orientation and configuration but should not correct for
small scale differences, as this will tend to make the atlases very
similar to each other and harder to rank with respect to a given query

By contrast, the propagation of atlases during multi-atlas segmentation
uses transformations with a high degree of local control which,
as discussed in Background: multi-atlas segmentation, have been
shown to generate more accurate segmentations. The non-rigid
registration method used to propagate atlas segmentations can be
chosen from among a number of different approaches (see Zitová and
Flusser (2003) for an overview). This work uses the free-form
deformation (FFD) model of Rueckert et al. (1999) where displacements
at a lattice of control points are blended using B-spline basis
functions (De Boor, 1978).

Another distinction needs to be made between the use of a
similarity metric for selection and its use for registration. During
registration, the similarity metric is used as an optimisation objective
function, whereas for selection, the similarity metric is evaluated once
post hoc for the query and atlas images after alignment. We have used
the same metric (NMI) for selection and for registrations (both pre-
and post-selection) although there is no strict requirement that the
same metric is used in all stages.

The type of spatial normalisation carried out has an effect on the
selection process. If the atlases and the query are only rigidly aligned
to the standard space, selection will favour atlases that are already
very similar in size and configuration to the query while some atlases
may be rejected that could have been useful for segmentation after, for
example, a global change of scale.

In contrast, high-resolution non-rigid normalisation implies that
the variation among atlas subjects is mainly represented in the
normalising transformations rather than the aligned images and an
image-based ranking of the atlases becomes harder to apply.

While it is possible to extract features from transformations as a
basis for selection (see for example Commowick and Malandain
(2007)), our focus in this work is on image similarity as a selection
criterion and an intermediate level of spatial normalisation is used.
These could be, for example, affine 12-parameter transformations or
coarse non-rigid transformations that only correct for large scale
configurational differences.

In terms of computation, most of the cost of multi-atlas segmentation
with image similarity selection is incurred by registrations.
If N atlases are in the database and $S$ of the top-ranked atlases
are propagated and fused,
then the number of registrations for a given query is $1+N+S$.
The $N$ registrations spatially normalising the atlases prior to selection can,
however, be carried out `off-line' so that the bulk of
the on-line computational cost is represented by
the $S$ fine-scale non-rigid registrations
that propagate the segmentations to the query.

\subsubsection*{Selection using meta-information}
\section*{Data and experiments}
\subsection*{Atlas database and choice of standard space}
\subsection*{Implementation}
\subsubsection*{Image similarity selection: assessing the accuracy obtained}
\subsubsection*{Image similarity as a selection criterion}
\subsubsection*{Varying the number of atlases selected}
\subsubsection*{Comparing similarity- and age-based selection}
\section*{Discussion}
\section*{Acknowledgments}
\section*{References}
