% !Mode:: "TeX:UTF-8"

\chapter*{Multi-atlas based segmentation of brain images:
Atlas selection and its effect on accuracy}
%\section*{Abstract}
%\section*{Introduction}
\section*{Methods}
%\subsection*{Background: multi-atlas segmentation}
\subsection*{Atlas selection}
\subsubsection*{Motivation and background}
The size of the atlas database can affect
various aspects of the process of multi-atlas segmentation
as well as the quality of the final segmentation.
As discussed in \emph{Background: multi-atlas segmentation},
the average segmentation accuracy 
achieved by fusing random sets of atlases increases asymptotically
as the number fused becomes large.
This asymptotic increase in accuracy means that
there are diminishing returns in using larger and larger numbers of atlases.
For a large atlas database, however,
the increased computational cost of registering
large numbers of atlases to the query image is an immediate practical problem.
A second difficulty relates to the way
in which an anatomical structure might vary across the population.
If a structure is represented by, say,
two morphologically distinct variants in the population
(and in the atlas database),
then fusing a large number of atlases may give a shape
that does not represent either variant very well.
In such circumstances, for a given query subject,
only a proper subset of the atlases in the database is appropriate to use
— those sharing the variant represented in the query.
Finally, in our experience,
the fusion of a large number of atlases
is more likely to create a smooth estimate of the structure
being segmented and yet a shape which is less smooth may be a better estimate.

For such reasons, the selection of a limited number of atlases,
appropriate for the query subject and prior to multi-atlas segmentation,
would appear preferable to the fusion of
an arbitrarily large number of atlases.
Furthermore, it is natural to ask
whether the asymptotic level of accuracy given by
fusing large numbers of random atlases is the best that can be achieved,
i.e. whether it is possible to equal or exceed this accuracy
by fusing a smaller number of selected atlases.

In previous work on atlas-based segmentation,
the term `selection' has typically been used to describe
the identification of the single best atlas for propagation to a query
(see, for example, Rohlfing et al. (2004a) and Wang et al. (2005)).
Han et al. (2008)
compare the selection of a single atlas against
the propagation and fusion of their entire atlas database.
Our earlier work (Aljabar et al., 2007) and
the work in this paper contrasts with these approaches
as we apply selection to whole sets of atlases prior
to propagation to the target and decision fusion.
Klein et al. (2008) use a similar approach
where multi-atlas segmentation is carried out
using only atlases that reach a user-defined threshold of similarity
with the target.
For a given threshold, the number of atlases used for different targets
may vary.
Our work contrasts this by fixing (for each experiment) the number of atlases
selected for different targets.
This enables an assessment of the effect of selection on segmentation quality
across a range of subjects.

The selection of atlases for segmenting a particular query image
also has parallels with clustering problems. For a large atlas database,
it is possible to search for clusters or modes of the population it
represents. Given an unseen query image and an estimate of the
cluster it belongs to, a better segmentation is expected using images
from the same cluster rather than from different ones. See Blezek and
Miller (2007) and Sabuncu et al. (2008) for examples of clustering
approaches applied to brain MR images.


\subsubsection*{Proposed methods for selection}
Given a large atlas database, a fixed subset size and a query image,
it is theoretically possible to identify the optimal subset of atlases for
generating a multi-atlas segmentation of the query. Leaving aside the
question of how segmentations given by different subsets can be
compared, exhaustively searching all possible subsets is clearly
impractical for a large database.

We therefore present an alternative heuristic approach that uses a
measure of the ‘similarity’ of each atlas to the query subject. Once
assigned, this measure can be used to rank the atlases. Multi-atlas
segmentation can then be carried out using a number of the top-ranked
atlases. A simple approach is to interpret similarity as image similarity,
i.e. to derive it from the intensities in the query and atlas images.
Alternatively, similarity may be derived from meta-information relating
to the subjects. The ranking of each atlas is then based on how closely
the atlas subject matches the query in terms of a clinical variable, such
as age, pathology, clinical history, genetics, gender, handedness, etc.

%\subsubsection*{Selection using image similarity}
%\subsubsection*{Selection using meta-information}
%\section*{Data and experiments}
%\subsection*{Atlas database and choice of standard space}
%\subsection*{Implementation}
%\subsubsection*{Image similarity selection: assessing the accuracy obtained}
%\subsubsection*{Image similarity as a selection criterion}
%\subsubsection*{Varying the number of atlases selected}
%\subsubsection*{Comparing similarity- and age-based selection}
%\section*{Discussion}
%\section*{Acknowledgments}
%\section*{References}
