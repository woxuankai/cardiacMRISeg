% !Mode:: "TeX:UTF-8"

\chapter*{基于多谱图方法分割大脑图像:图谱选择与其对准确性影响}
%\section*{概括}
%\section*{简介}
\section*{方法}
\subsection*{背景:多谱图分割}
%\subsection*{图谱选择}
\subsubsection*{动机与背景}
图谱数据集的大小能影响多图谱分割过程中的若干方面,也能影响最终分割结果的质量。
正如在 背景:多谱图分割 中讨论的,
通过融合任意图谱集合获得的平均分割准确度的和融合的图谱数目是不同步增加的。
这种不同步的增加意味着,使用越来越多的图谱的收益在减少。
然而,对于大型的图谱数据库,
由于向目标图像配准大量图谱的机选花销是一个立即显现的、实际的问题。
第二个困难在于,解剖结构在人群中可能有差异。
如果一个结构在群体(和图谱数据集)中有两个不同的形态,
那么大量图谱的融合可能得到一个不能很好表现任意一个形态的结构。
在这种情况下,对于一个已经给定的目标图像,
数据库中只有一个恰当的子集才适合使用---那些和目标图像的拥有共同形态的图谱。
最后,根据我们的经验,融合大量的图谱容易得到对待分割组织的一个平滑的估计,
而一个不太平滑的形状可能会是更好的估计。

处于上述原因,与融合任意大量的图谱相比,在多谱图分割之前,
选择适合被分割图图像的、有限数量的图谱,可能看起来更好一些。
此外,很自然的,人们会想问,通过大量融合图谱所能逼近的准确度是否是最好的,
即,是否可能通过融合较少数量选择后的图谱的方法,来达到或者超过这个极限。

在先前对多谱图分割的研究中,
数据``选择''通常用于指代选择选择最好的图谱来进行标签传播
(参考,例如Rohlfing等人,2004a和Wang等人,2005年)。
Han等人在2008发表的文章中,
比较了选择单个图谱标签传播和选择整个图谱数据集的效果。
在我们的早期工作(Aljabar等人2007)和这篇文章的工作中,
在标签传播并且融合之前我们对整个图谱集施以选择,并与这些方法进行对比。
Klein等人在2008发表的文章中使用了相似的方法,
他们只使用达到认为设定的与目标图像相似度阈值的图谱。
对于一个确定的阈值,对于不同的目标,用到的图谱的数量可能会有所变化。
与此形成对比的是,对于不同的目标,我们的工作中固定了选取的图谱的个数。
这使我们能在很多目标图像的范围内评价图谱选择对分割质量的效果。

为了分割特定的图像,对图谱的选择与聚类问题类似。
对于大的图谱数据库,探索其表征的聚类或者模式是有可能的。
给定一个没遇到过的待配准目标和对目标所属类的一个估计,
与从不同类的图谱相比,使用属于同一个类的图谱应该能获得更好的分割结果。
Blezek与Miller在2007发表的文章和Sabuncu等人在2008发的文章
就是聚类方法在大脑磁共振图像中应用的例子。

\subsubsection*{提出的图谱选择的方法}
给定一个大的图谱数据库,一个固定个数的子集和一幅待配准的图片,
理论上来说,能找出用于多图谱方法分割目标图像的最优图谱子集。
且不论如何比较不同子集得到的分割好坏,
详尽的遍历所有可能的子集显然是不现实的,如果图谱数据集规模很大的话。

因此,我们提出了一个启发式的替代方案:
我们利用对每个图谱与待配准图像的``相似性''的计算。
一旦定下计算``相似性''的方法,这个计算结果就能用来对图谱进行排序。
排序之后,多图谱方法就可以领用排名最高的图像来进行图像分割。
一个简单的测定``相似性''的方法,便是图像相似性测度,
即从图谱图像和待分割图像的灰度信息中得到``相似性''。
或者,``相似性''也能从图像的元信息中得到。
在这种方法中,``相似性''是从临床信息中得到的,
例如年龄,病理,临床史,基因,性别,手性等等。

%\subsubsection*{Selection using image similarity}
%\subsubsection*{Selection using meta-information}
%\section*{Data and experiments}
%\subsection*{Atlas database and choice of standard space}
%\subsection*{Implementation}
%\subsubsection*{Image similarity selection: assessing the accuracy obtained}
%\subsubsection*{Image similarity as a selection criterion}
%\subsubsection*{Varying the number of atlases selected}
%\subsubsection*{Comparing similarity- and age-based selection}
%\section*{Discussion}
%\section*{Acknowledgments}
%\section*{References}
