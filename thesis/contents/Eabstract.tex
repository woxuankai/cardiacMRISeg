% !Mode:: "TeX:UTF-8"

\begin{Eabstract}%
{medical image processing}%
{multi-atlas}%
{image registration}%
{image segmentation}%
{cardiac MRI}
  Cardiovascular disease poses a great threat to human health.
  Studies have shown that morphological structure of the left ventricle
  and cardiovascular disease
  (such as heart failure, ischemic heart disease, etc.)
  prediction and diagnosis are directly related.
  Nuclear magnetic resonance imaging (MRI), with its unique advantages,
  is popular in the current clinical diagnosis of heart disease.
  The wildspread use of MRI
  and the rapid development of imaging technology
  have resulted in massive amounts of data.
  At the same time,
  manual image segmentation brings great workload to clinicians.
  Therefore,
  the automatic segmentation of cardiac MRI 
  has become one of the current research focuses.
  As the heart movement is non-rigid,
  the morphological differences in different periods of are huge,
  and the contrast between heart tissues is low,
  the heart of the automatic segmentation has always been difficult.
  Multi-atlas method has made great success in the brain image segmentation.
  In this paper, according to the characteristics of the cardiac image,
  we made some slightly changes to the multi-atlas method,
  and successfully completed the left ventricular segmentation.
\end{Eabstract}
